\chapwithtoc{Úvod}
Standartní meteorologické stanice záměrně neměří teplotu ve vegetaci. Typicky měří podmínky ve volné atmosféře ve standartizované výšce nad zemí. Data z těchto stanic, které jsou navzájem velmi dobře porovnatelná, určují globálně používaná klimatická data. Data mimo vegetaci neodrážejí podmínky, kterým jsou organismy v nich žijící vystaveny i přes to, že lesní porosty pokrývají až jednu třetinu pevninského povrchu.



Následky globálního oteplování dopadají na lesní organismy jinak kvůli porostu, který ovlivňuje mikroklima, kterému jsou vystaveny. Změny v porostu pak můžou významně ovlivnit podmínky pod ním. Dále ovšem může jít i o vliv topografie jako například sklon povrchu, který ovlivňuje mikroklima bylliného patra. 

Tato práce analyzuje rozdíl mezi teplotami v bylinném patře a ve výšce $\SI{2}{m}$ a rozdíl mezi naměřenými teplotami se snaží vysvětlit pomocí meteorologických proměnných jako vítr, oblačnost, vlhkost, srážky atp. 


!!!Odstavec o tom co je mikroklima, mezoklima, makroklima + tabulka viz Geiger: Climate near the ground



Introduction should answer the following questions, ideally in this order:
\begin{enumerate}
\item What is the nature of the problem the thesis is addressing?
\item What is the common approach for solving that problem now?
\item How this thesis approaches the problem?
\item What are the results? Did something improve?
\item What can the reader expect in the individual chapters of the thesis?
\end{enumerate}

Expected length of the introduction is between 1--4 pages. Longer introductions may require sub-sectioning with appropriate headings --- use \texttt{\textbackslash{}section*} to avoid numbering (with section names like `Motivation' and `Related work'), but try to avoid lengthy discussion of anything specific. Any ``real science'' (definitions, theorems, methods, data) should go into other chapters.
\todo{You may notice that this paragraph briefly shows different ``types'' of `quotes' in TeX, and the usage difference between a hyphen (-), en-dash (--) and em-dash (---).}

It is very advisable to skim through a book about scientific English writing before starting the thesis. I can recommend by.
