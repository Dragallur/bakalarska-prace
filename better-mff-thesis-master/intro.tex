\chapwithtoc{Úvod}
Meteorologická měření jsou po celém světě prováděna na standardizovaných meteorologických stanicích. Data z těchto stanic slouží npaříklad k předpovědi počasí nebo modelování klimatu. Z tohoto důvodu je snaha, aby na měřené veličiny, jako například teplotu, nemělo vliv místní podmínky. Meteorologická stanice by se pro to neměla vyskytovat ve městě, kde je silný vliv tepelného ostrova nebo naopak v lese, kde stromy brání volnému proudění vzduchu. Existují ovšem i případy, kdy nás zajímá lokální klima, například můžeme chtít studovat vliv vegetace na podmínky, kterým jsou organismy vystaveny \parencite{ZellwegerFlorian2019Sdou}.

Vegetace, ať už louka, les nebo jiný biotop, významně ovlivňuje lokální klimatické podmínky. Nejvýznamnějším vlivem, který můžeme pozorovat napříč všemi většimi lesními porosty po celém světě, je snížení teplot o $\SI{1}{\celsius}$ až $\SI{4}{\celsius}$ vůči teplotám mimo lesní porost. Tento pokles je způsoben stíněním slunečního záření, transpirací flóry, ovlivněním proudění vzduchu a mnoha dalšími navzájem interagujícími prvky klimatické systému. Mezoklimatem a makroklimatem myslíme klimatické podmínky na měřítku jednotek až desítek kilometrů zatímco mikroklimatem měřítko centimetrů až jednotek metrů. Lesním mikroklimatem máme na mysli klimatické podmínky uvnitř lesního porostu jako například teploty, vlhkost vzduchu, rychlost větru a podobně. Každé lesní mikroklima má svá specifika daná typem porostu, roční dobou nebo například topografií a nadmořskou výškou.

Oteplování klimatu způsobuje změny, kterým se musí organismy rychle adaptovat. Bez porozumění vztahu mezi makroklimatem a mikroklimatem v lese nemůžeme pochopit měnící se podmínky, které organismy zažívají. V posledních letech významně přibývá pozornost u tématu mikroklimatu v lesních porostech. Moderní čidla dokážou po velmi dlouhou dobu automaticky měřit podmínky v lese bez pravidelné přítomnosti odborníka nebo odbornice \parencite{WildJan2019Caer}. Tímto vzniká velké množství dat, ke kterým v minulosti nebyl přístup a otevírají se nové možnosti vědeckého bádání. Velkou pozornost proto dostávají témata jako vliv topografie nebo vliv typu lesního porostu na mikroklima skrze ukazatele jako otevřenost porostu, množství stromů v okolí, ale také vzdálenost k okraji lesa \parencite{ZellwegerFlorian2019Sdou, predictingforestmicroclimate, snow_deFrenneForestMicroclimates, LindenmayerDavid2022Sard}. Menší důraz se pak klade na vliv meteorologických podmínek na mikroklima. Z tohoto důvodu se snažíme spojit teoretické poznatky o vlivu vegetace na mikroklima s tím co víme z mikrometeorologie a mikroklimatologie.

Cílem této práce je analyzovat rozdíl mezi teplotami naměřenými v lesním porostu ve výšce $\SI{2}{m}$ nad zemí a $\SI{15}{cm}$ resp. $\SI{0}{cm}$ nad zemí. Zájmovou oblastí je Národní park Šumava a Národní park Bavorský les. Rozdíl mezi těmito teplotami se budeme snažit vysvětlit pomocí meteorologických podmínek naměřených na nejbližších stanicích: výšky sněhové pokrývky, oblačnosti, půdní vlhkosti, množství srážek, rychlosti větru a insolaci. Nulová hypotéza je, že v dostupných datech nemají tyto prediktory vliv na rozdíl teplot v lesním porostu. Alternativní hypotéza je, že existuje vztah mezi prediktory a rozdílem teplot. Cílem této práce není pouze zkoumání zdali existuje vztah mezi meteorologickými podmínkami a teplotami ve vegetaci, ale také hledání důvodů pro ne/přítomnost tohoto vztahu pomocí rešerše relevantní literatury. Vztah mezi meteorologickými podmínkami a rozdílem teplot poblíž země a ve výšce $\SI{2}{m}$ nad zemí v lesním porostu, nám může pomoct porozumět dynamice mikroklimatu. V budoucnosti by bylo možné navázat hlubší analýzou a modelováním a například provádět interpolaci na místa bez čidel nebo extrapolovat lesní mikroklima do minulosti.

První kapitola obsahuje teoretickou část práce. V první části rozebíráme energetickou bilanci poblíž zemského povrchu a následně se díváme na to jakým způsobem je mikroklima v lese ovlivněno přítomností vegetace a jaký vliv má například topografie na denní průběh teplot. V další části je popsáno jak meteorologické podmínky ovlivňují teplotu blízko země a nakonec představujeme statistické metody, které jsou využité při analýze dat.

Ve druhé kapitole ukazujeme použitá data z meteorologických stanic a z čidel napříč Národními parky Šumava a Bavorský les. Následně na jednom modelu ilustrujeme, jakým způsobem budeme statisticky zpracovávat data a například zde řešíme nenormalitu dat, jejich časovou nebo prostorovou korelaci.

Ve třetí kapitole představujeme 32 modelů rozdělených podle toho jakým čidlem se zabýváme, a jestli analyzujeme maximální nebo minimální denní teploty. V několika podkapitolách následně provedeme srovnání modelů mezi sebou a intepretujeme tyto výsledky pomocí znalostí z mikroklimatologie a mikrometeorologie. V poslední části se budeme soustředit na nedostatky a možná pokračování v tomto tématu.

Introduction should answer the following questions, ideally in this order:
\begin{enumerate}
\item What is the nature of the problem the thesis is addressing?
\item What is the common approach for solving that problem now?
\item How this thesis approaches the problem?
\item What are the results? Did something improve?
\item What can the reader expect in the individual chapters of the thesis?
\end{enumerate}

Expected length of the introduction is between 1--4 pages. Longer introductions may require sub-sectioning with appropriate headings --- use \texttt{\textbackslash{}section*} to avoid numbering (with section names like `Motivation' and `Related work'), but try to avoid lengthy discussion of anything specific. Any ``real science'' (definitions, theorems, methods, data) should go into other chapters.
\todo{You may notice that this paragraph briefly shows different ``types'' of `quotes' in TeX, and the usage difference between a hyphen (-), en-dash (--) and em-dash (---).}

It is very advisable to skim through a book about scientific English writing before starting the thesis. I can recommend by.

%Standartní meteorologické stanice záměrně neměří teplotu ve vegetaci. Typicky měří podmínky ve volné atmosféře ve standartizované výšce nad zemí. Data z těchto stanic, které jsou navzájem velmi dobře porovnatelná, určují globálně používaná klimatická data. Data mimo vegetaci neodrážejí podmínky, kterým jsou organismy v nich žijící vystaveny i přes to, že lesní porosty pokrývají až jednu třetinu pevninského povrchu.
