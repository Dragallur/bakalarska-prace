

%%% Choose a language %%%

\newif\ifEN
%\ENtrue   % uncomment this for english
\ENfalse   % uncomment this for czech

%%% Configuration of the title page %%%

\def\ThesisTitleStyle{mff} % MFF style
%\def\ThesisTitleStyle{cuni} % uncomment for old-style with cuni.cz logo
%\def\ThesisTitleStyle{natur} % uncomment for nature faculty logo

\def\UKFaculty{Faculty of Mathematics and Physics}
%\def\UKFaculty{Faculty of Science}

\def\UKName{Charles University in Prague} % this is not used in the "mff" style

% Thesis type names, as used in several places in the title
\def\ThesisTypeTitle{\ifEN BACHELOR THESIS \else BAKALÁŘSKÁ PRÁCE \fi}
%\def\ThesisTypeTitle{\ifEN MASTER THESIS \else DIPLOMOVÁ PRÁCE \fi}
%\def\ThesisTypeTitle{\ifEN RIGOROUS THESIS \else RIGORÓZNÍ PRÁCE \fi}
%\def\ThesisTypeTitle{\ifEN DOCTORAL THESIS \else DISERTAČNÍ PRÁCE \fi}
\def\ThesisGenitive{\ifEN bachelor \else bakalářské \fi}
%\def\ThesisGenitive{\ifEN master \else diplomové \fi}
%\def\ThesisGenitive{\ifEN rigorous \else rigorózní \fi}
%\def\ThesisGenitive{\ifEN doctoral \else disertační \fi}
\def\ThesisAccusative{\ifEN bachelor \else bakalářskou \fi}
%\def\ThesisAccusative{\ifEN master \else diplomovou \fi}
%\def\ThesisAccusative{\ifEN rigorous \else rigorózní \fi}
%\def\ThesisAccusative{\ifEN doctoral \else disertační \fi}



%%% Fill in your details %%%

% (Note: \xxx is a "ToDo label" which makes the unfilled visible. Remove it.)
\def\ThesisTitle{Analýza vazby mezi teplotou vzduchu ve standardní výšce a v hladině bylinného patra v závislosti na meteorologických podmínkách}
\def\ThesisAuthor{Vojtěch Klimeš}
\def\YearSubmitted{2023}

% department assigned to the thesis
\def\Department{Katedra fyziky atmosféry}
% Is it a department (katedra), or an institute (ústav)?
\def\DeptType{katedra}

\def\Supervisor{doc. Mgr. Michal Žák, Ph.D.}
\def\SupervisorsDepartment{Katedra fyziky atmosféry}

% Study programme and specialization
\def\StudyProgramme{Fyzika}
\def\StudyBranch{FP}

\def\Dedication{%
Dedication. \xxx{It is nice to say thanks to supervisors, friends, family, book authors and food providers.}
}

\def\AbstractEN{%
The forest microclimate is different from the climate measured at weather stations. Vegetation and macroclimate influence conditions in the herbaceous layer. In this thesis, we investigate the relationship between the temperature difference at $\SI{2}{m}$ above and near the ground surface in forest stands in the Sumava and Bavarian Forest National Parks and meteorological conditions. We show that for selected meteorological conditions measured at the nearest stations there is a relationship with the temperature difference from the dataloggers, strongest for cloud cover and snow depth and weakest for precipitation and wind speed. We build on studies investigating the effect of vegetation on forest temperatures to complete the picture of forest microclimate functioning. This knowledge is key to understanding the impact of climate change and to further research using advanced modelling and generalisation to other habitats.
% ABSTRACT IS NOT A COPY OF YOUR THESIS ASSIGNMENT!
}

\usepackage{siunitx}

\def\AbstractCS{%
Lesní mikroklima je odlišné od klimatu měřeného na meteorologických stanicích. Vegetace a makroklima ovlivňuje podmínky v bylinném patře. V práci se zabýváme vztahem mezi rozdílem teplot ve výšce $\SI{2}{m}$ nad zemí a poblíž zemského povrchu v lesním porostu v národních parcích Šumava a Bavorský les a meteorologickými podmínkami. Ukážeme, že pro vybrané meteorologické podmínky měřené ná nejbližších stanicích existuje vztah s rozdílem teplot z dataloggerů, nejsilnější pro oblačnost a výšku sněhu a nejslabší pro srážky a rychlost větru. Navazujeme na studie zkoumající vliv vegetace na teploty v lesním porostu a doplňujeme tím obrázek o fungování lesního mikroklimatu. Tyto znalosti jsou klíčové pro porozumění dopadu klimatické změny a k dalšímu výzkumu pomocí pokročilého modelování a zobecnění na jiné biotopy.
}

% 3 to 5 keywords (recommended), each enclosed in curly braces.
% Keywords are useful for indexing and searching for the theses by topic.
\def\Keywords{%
{mikroklimatologie} {mikrometeorologie} {lesní klima}% {Národní park Šumava} {Národní park Bavorský les}
}

% If your abstracts are long and do not fit in the infopage, you can make the
% fonts a bit smaller by this setting. (Also, you should try to compress your abstract more.)
% Alternatively, consider increasing the size of the page by uncommenting the
% geometry modification in thesis.tex.
%\def\InfoPageFont{}
\def\InfoPageFont{\small}  %uncomment to decrease font size

\ifEN\relax\else
% If you are writing a czech thesis, you additionally need to fill in the
% english translation of the metadata here!
\def\ThesisTitleEN{Study of relationship between temperature measurement in standard level and herb layer according to meteorological conditions}
\def\DepartmentEN{Department of Atmospheric Physics}
\def\DeptTypeEN{Department}
\def\SupervisorsDepartmentEN{Department of Atmospheric Physics}
\def\StudyProgrammeEN{Physics}
\def\StudyBranchEN{FP}
\def\KeywordsEN{%
{microclimatology} {micrometeorology} {forest climate}% {National park Šumava} {National park Bayerischer Wald}
}
\fi
