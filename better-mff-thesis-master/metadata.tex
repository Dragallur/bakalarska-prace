

%%% Choose a language %%%

\newif\ifEN
%\ENtrue   % uncomment this for english
\ENfalse   % uncomment this for czech

%%% Configuration of the title page %%%

\def\ThesisTitleStyle{mff} % MFF style
%\def\ThesisTitleStyle{cuni} % uncomment for old-style with cuni.cz logo
%\def\ThesisTitleStyle{natur} % uncomment for nature faculty logo

\def\UKFaculty{Faculty of Mathematics and Physics}
%\def\UKFaculty{Faculty of Science}

\def\UKName{Charles University in Prague} % this is not used in the "mff" style

% Thesis type names, as used in several places in the title
\def\ThesisTypeTitle{\ifEN BACHELOR THESIS \else BAKALÁŘSKÁ PRÁCE \fi}
%\def\ThesisTypeTitle{\ifEN MASTER THESIS \else DIPLOMOVÁ PRÁCE \fi}
%\def\ThesisTypeTitle{\ifEN RIGOROUS THESIS \else RIGORÓZNÍ PRÁCE \fi}
%\def\ThesisTypeTitle{\ifEN DOCTORAL THESIS \else DISERTAČNÍ PRÁCE \fi}
\def\ThesisGenitive{\ifEN bachelor \else bakalářské \fi}
%\def\ThesisGenitive{\ifEN master \else diplomové \fi}
%\def\ThesisGenitive{\ifEN rigorous \else rigorózní \fi}
%\def\ThesisGenitive{\ifEN doctoral \else disertační \fi}
\def\ThesisAccusative{\ifEN bachelor \else bakalářskou \fi}
%\def\ThesisAccusative{\ifEN master \else diplomovou \fi}
%\def\ThesisAccusative{\ifEN rigorous \else rigorózní \fi}
%\def\ThesisAccusative{\ifEN doctoral \else disertační \fi}



%%% Fill in your details %%%

% (Note: \xxx is a "ToDo label" which makes the unfilled visible. Remove it.)
\def\ThesisTitle{Analýza vazby mezi teplotou vzduchu ve standardní výšce a v hladině bylinného patra v závislosti na meteorologických podmínkách}
\def\ThesisAuthor{Vojtěch Klimeš}
\def\YearSubmitted{2023}

% department assigned to the thesis
\def\Department{Katedra fyziky atmosféry}
% Is it a department (katedra), or an institute (ústav)?
\def\DeptType{katedra}

\def\Supervisor{doc. Mgr. Michal Žák, Ph.D.}
\def\SupervisorsDepartment{Katedra fyziky atmosféry}

% Study programme and specialization
\def\StudyProgramme{Fyzika}
\def\StudyBranch{FP}

\def\Dedication{%
V první řadě bych chtěl poděkovat panu doc. Mgr. Michalu Žákovi, Ph.D. za trpělivost a věnovaný čas při vedení této práce. Za poskytnutá data a konzultace patří velké díky Mgr. Martinovi Kopeckému Ph.D., doc. Ing. Janu Wildovi Ph.D. a jejich kolegyním a kolegům. Dále bych chtěl poděkovat Českému hydrometeorologickému ústavu za možnost použití dostupných dat z meteorologických stanic. Speciální poděkování patří mým nejbližším za pomoc a podporu při psaní práce.
}

\def\AbstractEN{%
The forest microclimate is different from the conditions measured at meteorological stations. Conditions in the herbaceous layer are influenced by both the surrounding vegetation and macroclimatic factors. In this thesis, we investigate the relationship between the temperature difference at $\SI{2}{m}$ and near the ground surface in a forest and meteorological conditions. Microclimatic data are measured using temperature sensors in the national parks Šumava and Bavarian Forest. We build on studies investigating the influence of vegetation on temperatures in forest stands to complete the picture of how forest microclimate functions. This knowledge is crucial for understanding the impact of climate change and for further research using advanced modelling and generalisation to other habitats. For selected meteorological conditions measured at the stations, a statistically significant relationship was found for the temperature difference, which is strongest for snow depth and weakest for precipitation.
}
\usepackage{siunitx}
\def\AbstractCS{%
Lesní mikroklima je odlišné od podmínek měřených na meteorologických stanicích. Podmínky v bylinném patře jsou ovlivněny jak okolní vegetací, tak makroklimatickými faktory. V práci zkoumáme vztah mezi rozdílem teplot ve výšce $\SI{2}{m}$ nad zemí a poblíž zemského povrchu v lesním porostu a meteorologickými podmínkami. Mikroklimatická data jsou měřena pomocí teplotních čidel v národních parcích Šumava a Bavorský les. Navazujeme na studie zkoumající vliv vegetace na teploty v lesním porostu a doplňujeme tím obrázek o fungování lesního mikroklimatu. Tyto znalosti jsou klíčové pro porozumění dopadu klimatické změny a k dalšímu výzkumu pomocí pokročilého modelování a zobecnění na jiné biotopy. Pro vybrané meteorologické podmínky měřené na stanicích byl zjištěn statisticky významný vztah pro rozdíl teplot, který je nejsilnější pro výšku sněhu a nejslabší pro srážky. 
}
% 3 to 5 keywords (recommended), each enclosed in curly braces.
% Keywords are useful for indexing and searching for the theses by topic.
\def\Keywords{%
{mikroklimatologie} {mikrometeorologie} {lesní klima}% {Národní park Šumava} {Národní park Bavorský les}
}

% If your abstracts are long and do not fit in the infopage, you can make the
% fonts a bit smaller by this setting. (Also, you should try to compress your abstract more.)
% Alternatively, consider increasing the size of the page by uncommenting the
% geometry modification in thesis.tex.
%\def\InfoPageFont{}
\def\InfoPageFont{\footnotesize}  %uncomment to decrease font size

\ifEN\relax\else
% If you are writing a czech thesis, you additionally need to fill in the
% english translation of the metadata here!
\def\ThesisTitleEN{Study of relationship between temperature measurement in standard level and herb layer according to meteorological conditions}
\def\DepartmentEN{Department of Atmospheric Physics}
\def\DeptTypeEN{Department}
\def\SupervisorsDepartmentEN{Department of Atmospheric Physics}
\def\StudyProgrammeEN{Physics}
\def\StudyBranchEN{FP}
\def\KeywordsEN{%
{microclimatology} {micrometeorology} {forest climate}% {National park Šumava} {National park Bayerischer Wald}
}
\fi
