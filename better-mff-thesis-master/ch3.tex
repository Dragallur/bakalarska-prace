\chapter{Výsledky a diskuze}\label{chap:ch3}
V první části této kapitoly ukážeme řadu modelů, jejichž výpočet byl po teoretické stránce popsán v kapitole \ref{chap:statistika}. V druhé části budeme diskutovat výsledky.

\section{Výsledky}
Nyní budeme studovat pomocí lineárního smíšeného modelu vztah mezi rozdílem maximální resp. minimálních teplot poblíž země a ve $\SI{2}{m}$ a meteorologických podmínek: výšce sněhu, půdní vlhkosti, oblačnosti, insolaci, rychlosti větru a množství srážek.

Uvedeme zde celkově 16 modelů. Nejdříve se budeme zabývat minimálními a maximálními denními teplotami naměřenými na čidlech blízko země a to ve výšce $\SI{15}{cm}$ nad zemí a $\SI{0}{cm}$ nad zemí. 

Dále rozdělíme data na dobu se sněhem a bez sněhu podle množství sněhu, které bylo naměřeno na stanici Churáňov \ref{fig:synop_snowcm}. Teplou sezónou (téměř bez sněhu) pak budeme nazývat období od začátku května do konce října a studená sezóna bude od začátku listopadu do konce dubna. Pro tyto dvě možnosti opět ukážeme modely pro maximální a minimální denní teploty naměřené v $\SI{15}{cm}$ a $\SI{0}{cm}$ nad zemí.

Poslední varianta se bude týkat pouze teplot v $\SI{15}{cm}$, protože změníme prediktor sníh na kategorickou proměnnou. Hodnota $0$ bude odpovídat hodnotám bez sněhu, $1$ hodnotám se sněhem menším než $\SI{15}{cm}$ a $2$ hodnotám nad $\SI{15}{cm}$.

Pro každý model uvedeme počet měření s kterými pracujeme, období z kterého data pochází, počet čidel, podmíněné a marginální $R^2$, koeficienty jednotlivých prediktorů v modelu, jejich chyba a označíme ty, které vyšly jako statisticky nevýznamné.

\subsection{Celosezónní modely}
V tabulce \ref{tab:basicmodels} jsou modely pro všechna dostupná data a kombinaci minimálních a maximálních denních teplot a výšky pozemního čidla $\SI{0}{cm}$ a $\SI{15}{cm}$.

\begin{table}
\centering\footnotesize\sf
\begin{tabular}{lrrrr}
\toprule
	& $\Delta t_{15cm,max}$ & $\Delta t_{0cm,max}$ & $\Delta t_{15cm,min}$ & $\Delta t_{0cm,min}$\\
\midrule
	Počet měření & $76627$ & $76635$ & $74225$ & $74083$\\
	Počet čidel & \multicolumn{4}{c}{157} \\
	$R_m^2$ & $0.031$ & $0.055$ & $0.14$ & $0.050$\\
	$R_c^2$ & $0.20$ & $0.16$ & $0.42$ & $0.28$\\
\midrule
	Konstanta & $\SI{0.42(6)}{}$ & $\SI{-0.28(7)}{}$ & $\SI{-1.23(5)}{}$ & $\SI{0.30(6)}{}$\\
	Výška sněhu & $\SI{0.0045(7)}{}$ & $\SI{0.0197(9)}{}$ & $\SI{0.0222(5)}{}$ & $\SI{0.0171(5)}{}$\\
	Oblačnost & $\SI{-0.041(8)}{}$ & $\SI{0.09(1)}{}$ & $\SI{0.211(6)}{}$ & $\SI{0.053(8)}{}$\\
	Vlhkost & $\SI{-0.6(1)}{}$ & $\SI{0.8(1)}{}$ & $\SI{1.9(1)}{}$ & $\SI{0.9(1)}{}$\\
	Srážky & \textcolor{gray}{$\SI{0.002(2)}{}$} & \textcolor{gray}{$\SI{0.02(1)}{}$} & \textcolor{gray}{$\SI{0.0006(6)}{}$} & $\SI{0.0008(4)}{}$\\
	Rychlost větru & $\SI{-0.0072(4)}{}$ & $\SI{-0.0138(5)}{}$ & $\SI{-0.0018(4)}{}$ & $\SI{-0.0184(5)}{}$\\
	Insolace & $\SI{0.00042(1)}{}$ & $\SI{-0.00026(1)}{}$ & $\SI{0.00006(2)}{}$ & $\SI{-0.00018(3)}{}$\\
\bottomrule
\end{tabular}
	\caption{Srovnání jednotlivých modelů pro všechna dostupná data. Šedě označené jsou hodnoty, pro které vyšla v F testu p hodnota $>0.05$, a nepovažujeme je statisticky významné od nuly (nezavrhli jsme nulovou hypotézu). Standartní chyba koeficientu je daná v závorce.}
\label{tab:basicmodels}
\end{table}

\subsection{Sezónní modely}
Podobně jako pro modely se všemi daty se nyní omezíme na teplé období (bez sněhu), tedy od května do října. Výsledky můžeme vidět v tabulce \ref{tab:warmhalfmodels}. Pro tyto modely jsme vyřadili prediktor výška sněhu.

\begin{table}
\centering\footnotesize\sf
\begin{tabular}{lrrrr}
\toprule
	& $\Delta t_{15cm,max}$ & $\Delta t_{0cm,max}$ & $\Delta t_{15cm,min}$ & $\Delta t_{0cm,min}$\\
\midrule
	Počet měření & $32062$ & $32108$ & $30716$ & $30651$\\
	Počet čidel & \multicolumn{4}{c}{157} \\
	$R_m^2$ & $0.098$ & $0.20$ & $0.24$ & $0.21$\\
	$R_c^2$ & $0.51$ & $0.60$ & $0.71$ & $0.57$\\
\midrule
	Konstanta & $\SI{-0.55(7)}{}$ & \textcolor{gray}{$\SI{-2.5(1)}{}$} & $\SI{-2.44(7)}{}$ & $\SI{-1.60(8)}{}$\\
	Oblačnost & $\SI{-0.16(1)}{}$ & $\SI{0.11(2)}{}$ & $\SI{0.265(7)}{}$ & $\SI{0.21(1)}{}$\\
	Vlhkost & $\SI{2.2(1)}{}$ & $\SI{6.7(2)}{}$ & $\SI{4.5(1)}{}$ & $\SI{5.1(2)}{}$\\
	Srážky & $\SI{-0.04(1)}{}$ & $\SI{0.05(1)}{}$ & \textcolor{gray}{$\SI{0.0005(6)}{}$} & $\SI{0.0059(9)}{}$\\
	Rychlost větru & $\SI{-0.0034(7)}{}$ & $\SI{-0.0020(9)}{}$ & $\SI{0.0106(5)}{}$ & $\SI{-0.0065(7)}{}$\\
	Insolace & $\SI{0.00065(1)}{}$ & $\SI{-0.00027(2)}{}$ & $\SI{0.00028(2)}{}$ & $\SI{0.00015(4)}{}$\\
\bottomrule
\end{tabular}
	\caption{Srovnání jednotlivých modelů pro teplé období. Šedě označené jsou hodnoty, pro které vyšla v F testu p hodnota $>0.05$, a nepovažujeme je statisticky významné od nuly (nezavrhli jsme nulovou hypotézu). Standartní chyba koeficientu je daná v závorce.}
\label{tab:warmhalfmodels}
\end{table}

V tabulce \ref{tab:coldhalfmodels} jsou výsledky z modelování dat pro studenou část roku, od listopadu do dubna.

\begin{table}
\centering\footnotesize\sf
\begin{tabular}{lrrrr}
\toprule
	& $\Delta t_{15cm,max}$ & $\Delta t_{0cm,max}$ & $\Delta t_{15cm,min}$ & $\Delta t_{0cm,min}$\\
\midrule
	Počet měření & $44563$ & $44528$ & $43505$ & $43429$\\
	Počet čidel & \multicolumn{4}{c}{156} \\
	$R_m^2$ & $0.066$ & $0.087$ & $0.12$ & $0.088$\\
	$R_c^2$ & $0.19$ & $0.16$ & $0.29$ & $0.23$\\
\midrule
	Konstanta & $\SI{0.96(7)}{}$ & $\SI{0.99(7)}{}$ & $\SI{-0.11(6)}{}$ & $\SI{1.63(6)}{}$\\
	Výška sněhu & $\SI{0.0031(7)}{}$ & $\SI{0.0158(8)}{}$ & $\SI{0.0214(6)}{}$ & $\SI{0.0152(6)}{}$\\
	Oblačnost & $\SI{0.03(1)}{}$ & $\SI{0.09(1)}{}$ & $\SI{0.185(9)}{}$ & $\SI{-0.05(1)}{}$\\
	Vlhkost & $\SI{-2.4(2)}{}$ & $\SI{-2.6(2)}{}$ & $\SI{-0.6(1)}{}$ & $\SI{-1.9(1)}{}$\\
	Srážky & \textcolor{gray}{\SI{0.003(2)}{}} & $\SI{-0.04(2)}{}$ & \textcolor{gray}{$\SI{-0.001(1)}{}$} & \textcolor{gray}{$\SI{-0.0002(4)}{}$}\\
	Rychlost větru & $\SI{-0.0098(6)}{}$ & $\SI{-0.0161(7)}{}$ & $\SI{-0.0073(5)}{}$ & $\SI{-0.0241(6)}{}$\\
	Insolace & $\SI{0.00029(2)}{}$ & $\SI{-0.00030(2)}{}$ & $\SI{-0.00027(4)}{}$ & $\SI{-0.00033(5)}{}$\\
\bottomrule
\end{tabular}
	\caption{Srovnání jednotlivých modelů pro studené období. Šedě označené jsou hodnoty, pro které vyšla v F testu p hodnota $>0.05$, a nepovažujeme je statisticky významné od nuly (nezavrhli jsme nulovou hypotézu). Standartní chyba koeficientu je daná v závorce.}
\label{tab:coldhalfmodels}
\end{table}

\subsection{Modely pro sníh jako kategorickou proměnnou}
Jako poslední sadu modelů, ukážeme ty pro které, jsme výšku sněhu nahradili kategorickou proměnnou. Celkově spočítáme čtyři modely, které budou kombinací maximálních a minimálních teplot a období se sněhem a všech dat, abychom mohli modely lépe srovnávat.

\begin{table}
\centering\footnotesize\sf
\begin{tabular}{lrrrr}
\toprule
	& $\Delta t_{all,max}$ & $\Delta t_{all,min}$ & $\Delta t_{cold,max}$ & $\Delta t_{cold,min}$\\
\midrule
	Počet měření & $76627$ & $74225$ & $44563$ & $43505$\\
	Počet čidel & \multicolumn{2}{c}{157} & \multicolumn{2}{c}{156}\\
	$R_m^2$ & $0.032$ & $0.090$ & $0.067$ & $0.055$\\
	$R_c^2$ & $0.20$ & $0.37$ & $0.19$ & $0.22$\\
\midrule
	Konstanta & $\SI{0.43(6)}{}$ & $\SI{-1.17(5)}{}$ & $\SI{0.99(7)}{}$ & $\SI{-0.1(6)}{}$\\
	Výška sněhu & $\SI{0.040(9)}{}$ & $\SI{0.226(7)}{}$ & \textcolor{gray}{$\SI{0.005(9)}{}$} & $\SI{0.211(8)}{}$\\
	Oblačnost & $\SI{-0.040(8)}{}$ & $\SI{0.217(6)}{}$ & $\SI{0.03(1)}{}$ & $\SI{0.194(9)}{}$\\
	Vlhkost & $\SI{-0.6(1)}{}$ & $\SI{1.8(1)}{}$ & $\SI{-2.4(2)}{}$ & $\SI{-0.7(1)}{}$\\
	Srážky & \textcolor{gray}{$\SI{0.002(2)}{}$} & \textcolor{gray}{$\SI{0.0005(6)}{}$} & \textcolor{gray}{$\SI{0.003(2)}{}$} & \textcolor{gray}{$\SI{-0.002(1)}{}$}\\
	Rychlost větru & $\SI{-0.0072(4)}{}$ & $\SI{-0.0016(4)}{}$ & $\SI{-0.0098(6)}{}$ & $\SI{-0.0071(5)}{}$\\
	Insolace & $\SI{0.00042(1)}{}$ & $\SI{0.00006(2)}{}$ & $\SI{0.00028(2)}{}$ & $\SI{-0.00026(4)}{}$\\
\bottomrule
\end{tabular}
	\caption{Srovnání jednotlivých modelů, pro které nahradíme mocnost sněhu kategorickou proměnnou. Indexy $all$ a $cold$ označujeme, jestli jde o o model používající všechna data nebo pouze pro období roku se sněhem. Šedě označené jsou hodnoty, pro které vyšla v F testu p hodnota $>0.05$, a nepovažujeme je statisticky významné od nuly (nezavrhli jsme nulovou hypotézu). Standartní chyba koeficientu je daná v závorce.}
	\label{tab:snowcategoricalmodels}
\end{table}

\section{Diskuze}
V první části diskuze se podíváme na jednotlivé modely s budeme je srovnávat mezi sebou. V druhé části se zaměříme na jednotlivé prediktory a budeme diskutovat jejich vliv na rozdíl teplot.

\subsection{Srovnání modelů pro maximální teploty a výšku $\SI{15}{cm}$}
V tabulce \ref{tab:max15cm_models} srovnáváme modely pro maximální denní teploty a výšku $\SI{15}{cm}$ mezi sebou. Když srovnáme kvalitu modelů jako množství vysvětlené variability tak model pro teplou sezónu vychází nejlépe s $R_c^2 = 0.51$. Z výsledků můžeme usuzovat, že přítomnost sněhu ovlivňuje výrazně vliv ostatních prediktorů. Pokud srovnáme druhý a třetí model, vidíme že oblačnost má opačný vliv na rozdíl teplot. Zatímco v teplé sezóně znamená přítomnost oblačnosti nižší míru ohřevu pozemního čidla a tedy menší rozdíl teplot tak v zimě se situace komplikuje. Jestliže je čidlo nad sněhovou pokrývkou tak opět ohřev během dne je menší, ale když je pod sněhem tak se téměř neohřívá a často nastávají maximální teploty v noci. Poté větší oblačnost způsobuje menší ztrátu tepla dlouhovlnným zářením pro čidlo ve $\SI{2}{m}$, ovšem ta je mitigovaná i tím, že se nacházíme v lese, kde je pokles teplot v noci menší než mimo něj. Oblačnost také souvisí v zimě s nárůstem sněhové pokrývky, která má kladný vliv na rozdíl teplot. Vliv těchto efektů způsobuje, že ve studeném období oblačnost zvyšuje rozdíl teplot. Všimněme si, že vliv je několika násobně menší než v teplém období a má vysokou relativní chybu ($\SI{30}{\%}$).

Dále můžeme pozorovat, že vliv půdní vlhkosti na rozdíl teplot je v teplém období kladný a ve studeném záporný. Pro celou časovou řadu pak vychází lehce záporný, což je způsobeno, tím že studené období má celkově více měření. Opět pokud je na zemi přítomný sníh půdní vlhkost může rychle vzrůst, když stoupnou teploty a sníh začne tát. Pokud je čidlo stále pod sněhem (nebo hned nad ním a sníh ho stále ochlazuje) tak ovšem budou teploty blízké $\SI{0}{\celsius}$, zatímco ve výšce $\SI{2}{m}$ můžou teploty růst více. Oproti stavu, kdy sníh netál, protože teploty byly záporné se teploty ve $\SI{2}{m}$ prudce zvýší a tím dojde k poklesu rozdílu teplot, jak vidíme v tabulce pro studené období.

\begin{table}
\centering\footnotesize\sf
\begin{tabular}{lrrrrr}
\toprule
	& $\Delta t$ & $\Delta t$ & $\Delta t$ & $\Delta t_{cat}$ & $\Delta t_{cat}$\\
\midrule
	Počet měření & $76627$ & $32062$ & $44563$ & $76627$ & $44563$\\
	Období & vše & teplé & studené & vše & studené \\
	Počet čidel & $157$ & $157$ & $156$ & $157$ & $156$ \\
	$R_m^2$ & $0.031$ & $0.098$ & $0.066$ & $0.032$ & $0.067$\\
	$R_c^2$ & $0.2$ & $0.51$ & $0.19$ & $0.20$ & $0.19$\\
\midrule
	Konstanta & $\SI{0.42(6)}{}$ & $\SI{-0.55(7)}{}$ & $\SI{0.96(7)}{}$ & $\SI{0.43(6)}{}$ & $\SI{0.99(7)}{}$\\
	Výška sněhu & $\SI{0.0045(7)}{}$ & - & $\SI{0.0031(7)}{}$ & $\SI{0.040(9)}{}$ & \textcolor{gray}{$\SI{0.005(9)}{}$}\\
	Oblačnost & $\SI{-0.041(8)}{}$ & $\SI{-0.16(1)}{}$ & $\SI{0.03(1)}{}$ & $\SI{-0.040(8)}{}$ & $\SI{0.03(1)}{}$\\
	Vlhkost & $\SI{-0.6(1)}{}$ & $\SI{2.2(1)}{}$ & $\SI{-2.4(2)}{}$ & $\SI{-0.6(1)}{}$ & $\SI{-2.4(2)}{}$\\
	Srážky & \textcolor{gray}{$\SI{0.002(2)}{}$} & $\SI{-0.04(1)}{}$ & \textcolor{gray}{$\SI{0.003(2)}{}$} & \textcolor{gray}{$\SI{0.002(2)}{}$} & \textcolor{gray}{$\SI{0.003(2)}{}$}\\
	Rychlost větru & $\SI{-0.0072(4)}{}$ & $\SI{-0.0034(7)}{}$ & $\SI{-0.0098(6)}{}$ & $\SI{-0.0072(4)}{}$ &$\SI{-0.0098(6)}{}$\\
	Insolace & $\SI{0.00042(1)}{}$ & $\SI{0.00065(1)}{}$ & $\SI{0.00029(2)}{}$ & $\SI{0.00042(1)}{}$ & $\SI{0.00028(2)}{}$\\
\bottomrule
\end{tabular}
	\caption{Srovnání modelů pro maximální denní teploty a výšku $\SI{15}{cm}$. Indexem $cat$ označujeme, že jsme výšku sněhu nahradili kategorickou proměnnou. Šedě označené jsou hodnoty, pro které vyšla v F testu p hodnota $>0.05$, a nepovažujeme je statisticky významné od nuly (nezavrhli jsme nulovou hypotézu). Standartní chyba koeficientu je daná v závorce.}
	\label{tab:max15cm_models}
\end{table}

\subsection{Nahrazení oblačnosti pomocí reanalýzy ERA5}

Instead, try some of the following:
\begin{itemize}
\item State a hypothesis and prove it statistically
\item Show plots with measurements that you did to prove your results (e.g. speedup). Use either \texttt{R} and \texttt{ggplot}, or Python with \texttt{matplotlib} to generate the plots.\footnote{Honestly, the plots from \texttt{ggplot} look \underline{much} better.} Save them as PDF to avoid printing pixels (as in \cref{fig:f}).
\item Compare with other similar software/theses/authors/results, if possible
\item Show example source code (e.g. for demonstrating how easily your results can be used)
\item Include a `toy problem' for demonstrating the basic functionality of your approach and detail all important properties and results on that
\item Include clear pictures of `inputs' and `outputs' of all your algorithms, if applicable
\end{itemize}

\section{What is a discussion?}
\begin{itemize}
\item What is the potential application of the result?
\item Does the result solve a problem that other people encountered?
\item Did the results point to any new (surprising) facts?
\item How (and why) is the approach you chose different from what the others have done previously?
\item Why is the result important for your future work (or work of anyone other)?
\item Can the results be used to replace (and improve) anything that is used currently?
\end{itemize}

If you do not know the answers, you may want to ask the supervisor. Also, do not worry if the discussion section is half-empty or completely pointless; you may remove it completely without much consequence. It is just a bachelor thesis, not a world-saving avenger thesis.
