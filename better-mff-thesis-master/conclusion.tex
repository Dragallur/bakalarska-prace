\chapwithtoc{Závěr}
V úvodu jsme zformulovali nulovou hypotézu a alternativní hypotézu. Pomocí 32 lineární smíšených modelů jsme nulovou hypotézu vyvrátili a zjistili, že mezi rozdílem maximálních resp. minimálních denních teplot na čidlech ve výšce $\SI{0}{cm}$ resp. $\SI{15}{cm}$ a $\SI{2}{m}$ a meteorologickými proměnnými existuje vztah, neboli jejich koeficienty jsou pro většinu modelů nenulové. Tento vztah jsme zkoumali u řady prediktorů: výška sněhu, oblačnost, půdní vlhkost, množství srážek, rychlost větru a insolace. Nejslabším prediktorem bylo množství srážek, které pro většinu modelů z kategorie "bez absolutní hodnoty" vycházelo neprůkazně, tedy že zde vztah neexistuje. Pro modely "s absolutní hodnotou" ovšem v několika případech vycházel prediktor statisticky signifikantní. U všech ostatních prediktorů jsme s vysokou jistotou ukázali, že na rozdíl teplot mají vliv.

Kromě statistického zpracování dat jsme také provedli interpretaci výsledků v kontextu rešerše našich znalostí z oblasti mikroklimatologie a mikrometeorologie. Ojediněle jsme nedokázali některé výsledky s jistotou vysvětlit, hlavním důvodem zde byla nelokálnost měření meteorologických podmínek. Nebylo zde tedy v některých případech možné udělat hlubší analýzu problematiky kvůli kvalitě dat.

Dále jsme navrhli způsob, jak výsledky vylepšit. Kvality dat by byla výrazně vylepšena měřením meteorologických podmínek i v lesním porostu, ale také prodloužením období, kterým jsme se zabývali, tak bychom mohli některé prediktory upravit nebo nahradit jinými, například insolaci nahradit hodinovým úhlem a maximální denní insolací. Ze statistického hlediska by největším zlepšením zřejmě bylo vytvořit složitější prostoročasový model. Výsledky práce otevírají možnosti hlubšímu výzkumu, který by mohl vést například k interpolaci chybějících mikroklimatických dat a pomáhají nám porozumět interakci mezi teplotami v lesním porostu a ostatními meteorologickými podmínkami.

In the conclusion, you should summarize what was achieved by the thesis. In a few paragraphs, try to answer the following:
\begin{itemize}
\item Was the problem stated in the introduction solved? (Ideally include a list of successfully achieved goals.)
\item What is the quality of the result? Is the problem solved for good and the mankind does not need to ever think about it again, or just partially improved upon? (Is the incompleteness caused by overwhelming problem complexity that would be out of thesis scope\todo{This is quite common.}, or any theoretical reasons, such as computational hardness?)
\item Does the result have any practical applications that improve upon something realistic?
\item Is there any good future development or research direction that could further improve the results of this thesis? (This is often summarized in a separate subsection called `Future work'.)
\end{itemize}
