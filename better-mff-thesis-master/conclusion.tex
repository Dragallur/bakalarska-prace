\chapwithtoc{Závěr}
V úvodu jsme zformulovali nulovou hypotézu a alternativní hypotézu. Pomocí 32 lineární smíšených modelů jsme nulovou hypotézu vyvrátili a zjistili, že mezi rozdílem maximálních resp. minimálních denních teplot na čidlech ve výšce $\SI{0}{cm}$ resp. $\SI{15}{cm}$ a $\SI{2}{m}$ a meteorologickými proměnnými ze staničních měření existuje vztah, neboli jejich koeficienty jsou pro většinu modelů nenulové. Tento vztah jsme zkoumali u řady prediktorů: výška sněhu, oblačnost, půdní vlhkost, množství srážek, rychlost větru a insolace. Nejslabším prediktorem bylo množství srážek, které pro většinu modelů z kategorie "bez absolutní hodnoty" vycházelo neprůkazně, tedy že zde vztah neexistuje. Pro modely "s absolutní hodnotou" ovšem v několika případech vycházel prediktor statisticky signifikantní. U všech ostatních prediktorů jsme s vysokou jistotou ukázali, že na rozdíl teplot mají vliv.

Kromě statistického zpracování dat jsme také provedli interpretaci výsledků v kontextu rešerše našich znalostí z oblasti mikroklimatologie a mikrometeorologie. Ojediněle jsme nedokázali některé výsledky s jistotou vysvětlit, hlavním důvodem zde byla relativní vzdálenost čidel od meteorologických stanic, které měřily meteorologické podmínky. Nebylo zde tedy v některých případech možné udělat hlubší analýzu problematiky kvůli kvalitě dat.

Dále jsme navrhli způsob, jak výsledky vylepšit. Kvality dat by byla výrazně vylepšena měřením meteorologických podmínek i v lesním porostu, ale také zpracováním delšího časového období z následujících let. Tato data v době zpracovávání nebyla dostupná. Některé prediktory bychom mohli upravit nebo nahradit jinými. Například insolaci nahradit hodinovým úhlem a maximální denní insolací. Ze statistického hlediska by největším zlepšením zřejmě bylo vytvořit složitější prostoročasový model. Výsledky práce otevírají možnosti hlubšímu výzkumu, který by mohl vést například k interpolaci chybějících mikroklimatických dat a pomáhají nám porozumět interakci mezi teplotami v lesním porostu a ostatními meteorologickými podmínkami.
