\chapwithtoc{Závěr}
V této práci jsme zkoumali vliv meteorologických podmínek na rozdíl teplot poblíž zemského povrchu v lesním porostu. K tomu jsme využili data z meteorologických stanic a ze $157$ čidel rozmístěných napříč národními parky Šumava a Bavorský les. Zpracovávali jsme dostupná data od 12.10.2019 do 17.5.2021. Spočítali jsme 32 lineární smíšených modelů rozdělených do hlavních skupin podle toho, jestli závislá proměnná byla v absolutní hodnotě (modely s absolutní hodnotou), jestli jsme studovali maximální nebo minimální teploty, a jestli zdroj teplot u země bylo čidlo ve výšce $\SI{0}{cm}$ nebo $\SI{15}{cm}$. Nulovou hypotézu jsme pomocí modelů vyvrátili a zjistili, že mezi rozdílem maximálních, resp. minimálních denních teplot na čidlech ve výšce $\SI{0}{cm}$ resp. $\SI{15}{cm}$ a $\SI{2}{m}$ a meteorologickými proměnnými ze staničních měření existuje vztah, neboli jejich koeficienty jsou pro většinu modelů nenulové. Tento vztah jsme zkoumali u řady prediktorů: výška sněhu, oblačnost, půdní vlhkost, množství srážek, rychlost větru a insolace.

Kromě statistického zpracování dat jsme také provedli interpretaci výsledků v kontextu rešerše našich znalostí z oblasti mikroklimatologie a mikrometeorologie. Ukázali jsme, že výška sněhu má kladný vliv na rozdíl teplot a nahrazení výšky sněhu za kategorickou proměnnou (žádný sníh, čidlo nad sněhem a čidlo pod sněhem) může výrazně zvýšit sílu prediktoru. Jako hlavní vliv výšky sněhu jsme určili to, že pod dostatečnou vrstvou sněhu se teplota pohybuje okolo $\SI{0}{\celsius}$. Prediktor oblačnosti má silný záporný vliv na rozdíl teplot, především protože při vyšší oblačnosti dochází k menšímu ohřevu povrchu. Rychlost větru má záporný vliv na rozdíl teplot, protože dochází k silnějšímu promíchávání vzduchu. Insolace má slabý kladný vliv, protože souvisí s ohřevem povrchu. Množství srážek byl nejméně průkazný prediktor, který pro některé modely ukazoval pokles absolutního rozdílu teplot, který byl silnější pro čidla ve výšce $\SI{0}{cm}$. Půdní vlhkost byla silným prediktorem, který se ovšem výrazně lišil mezi modely s a bez absolutní hodnoty a mezi jednotlivými částmi roku. Možné důvody jsme uvedli v diskuzi. Hlavním důvodem pro nejistotu v interpretaci všech prediktorů byla relativní vzdálenost čidel od meteorologických stanic, které měřily meteorologické podmínky. Nebylo zde tedy v některých případech možné udělat hlubší analýzu problematiky kvůli kvalitě dat.

Dále jsme navrhli způsob, jak možnosti studia mikroklimatu vylepšit. Kvalita analýzy by byla výrazně vylepšena měřením meteorologických podmínek i v lesním porostu, ale také zpracováním delšího časového období z následujících let. Některé prediktory bychom mohli v hlubší analýze upravit nebo nahradit jinými. Ze statistického hlediska by největším zlepšením zřejmě bylo vytvořit složitější prostoročasový model. Výsledky práce otevírají možnosti dalšímu výzkumu, který by mohl vést například k interpolaci chybějících mikroklimatických dat, a pomáhají nám porozumět interakci mezi teplotami v lesním porostu a ostatními meteorologickými podmínkami.
