\chapter{Analýza lesního mikroklimatu}
\label{chap:ch1}

V následující části \ref{chap:fyz} popíšeme fyzikální děje odehrávající se poblíž zemského povrchu z pohledu mikrometeorologie a mikroklimatologie. Od jednoduchých ilustračních příkladů se přesuneme k tomu jaký vliv má na mikroklima přítomnost vegetace \ref{chap:veg} a topografie \ref{chap:topo}. V závěřečné části této kapitoly \ref{chap:measure} bude popsáno jakým způsobem jsou meteorologické podmínky měřeny a nakonec v \ref{chap:sumavabavorskyles} se podíváme na klima typické pro Národní park Šumava a Bavorský les.

\section{Fyzikální pohled na děje při povrchu země} \label{chap:fyz}
Pro rovný a téměř homogenní povrch, který můžeme omezit zeshora a zespoda rovinnou, platí zjednodušenná rovnice. Tato rovnice vyjadřuje energetickou bilanci\cite{arya2001}

$$R_N = H + H_L + H_G + \Delta H_S,$$\label{eq:bilance}

kde $R_N$ je celková bilance záření, $H$ je tok tepla z nebo do atmosféry, $H_L$ je latentní teplo, $H_G$ je tok tepla ze země a $\Delta H_S$ je změna uchovaného tepla za jednotku času na jednotku plochy přes celou hloubku vrstvy. $\Delta H_S$ můžeme také interpretovat jako rozdíl mezi energií dodanou a odevzdanou z vrstvy která nás zajímá. Pak jestliže $\Delta H_S>0$ tak se vrstva ohřívá a pro $\Delta H_S<0$ tak se ochlazuje jelikož $H_{in}<H_{out}$. Většinou ovšem nemůžeme měřit přímo toky tepla, pokud nejde o jednoduchý příklad jako třeba na obrázku (+++), kde vidíme vývoj toku energie na povrchu vyschlého jezera. Přes den se povrch ohřívá díky dopadu slunečního záření $R_N>0$, část tepla uniká pod povrch tedy $H_G<0$ a část ohřívá atmosféru, která sama pouze velmi málo absorbuje záření $H<0$. Naopak v noci vyzařuje povrch dlouhovlnné záření a tím se ochlazuje tedy $R_N<0$, $H$ je téměř nulové a zahřátá půda zpětně ohřívá povrch tedy $H_G>0$. Po celou dobu jde o suchý povrch tedy $H_L=0$\cite{arya2001}.

\subsection{Vliv latentního tepla}
Latentní teplo, které značíme $H_L$, je veličina týkající se změny skupenství látek, v našem případě se týká vody. Formálně můžeme latentní teplo zavést jako $H_L = T\Delta s$, kde $T$ je teplota při které dochází ke změně skupenství a $\Delta s$ je rozdíl mezi molárními entropiemi obou fází\cite{callen1985}. Za standartního atmosférického tlaku $\SI{1013}{hPa}$ je měřné latentní teplo tání $H_{lT} = \SI{334}{kJ/kg}$ a latentní teplo vypařování $H_{lv} = \SI{2265}{kJ/kg}$. 

Výše jsme analyzovali situaci, kdy šlo o suchý povrch. Pro vlhký povrch se nám situace změní. Část tepla bude absorbována vodou a spotřebována na výpar. Můžeme ovšem mít i opačnou situaci, kdy dochází ke kondenzaci vodní páry a uvolňování latentního tepla. Analogicky můžeme uvažovat situaci v zimě, kdy je přítomná voda v pevném skupenství. V našem zjednodušenném příkladě je tedy přes den $H_L > 0$ a v noci $H_L < 0$. Schématicky to můžeme vidět na \ref{fig:schema}, kde velikost jednotlivých toků tepla je ovlivněna mnoha faktory\cite{arya2001}.

\begin{figure}
\centering
	\begin{tikzpicture}
  % Part a: Day
  \draw (-1,0) -- (4,0);
  \draw[pattern=north east lines] (-1,0) rectangle (4,-0.2);
	\draw[-{Stealth},thick] (1,3) -- (1,0) node[midway,left] {$R_N$};
  \draw[-{Stealth},thick] (2,0) -- (2,2) node[midway,right] {$H$};
  \draw[-{Stealth},thick] (3,0) -- (3,1) node[midway,right] {$H_L$};
  \draw[-{Stealth},thick] (0,-0.2) -- (0,-1) node[midway,left] {$H_G$};
  \node at (2,-2) {(a) Situace ve dne};
	\node at (-0.7,0.25) {\textit{Povrch země}};

  % Part b: Night
  \begin{scope}[xshift=6cm]
    \draw (-1,0) -- (4,0);
    \draw[pattern=north east lines] (-1,0) rectangle (4,-0.2);
    \draw[-{Stealth},thick] (1,0) -- (1,1.5) node[midway,left] {$R_N$};
    \draw[-{Stealth},thick] (2,0.75) -- (2,0) node[midway,right] {$H$};
    \draw[-{Stealth},thick] (3,0.5) -- (3,0) node[midway,right] {$H_L$};
    \draw[-{Stealth},thick] (0,-1) -- (0,-0.2) node[midway,left] {$H_G$};
    \node at (2,-2) {(b) Situace v noci};
  \end{scope}
	\end{tikzpicture}
\caption{Schéma toku tepla ve dne a v noci}
\label{fig:schema}
\end{figure}

Vliv vlhkosti a deště můžeme ilustrovat na takzvaném "oázovém efektu". V určitých situacích, se může hodnota $H_L$ stát v rovnici \eqref{eq:bilance} i dominantní složkou. Oázovým efektem nazýváme situaci, kdy vítr přináší suchý teplý vzduch přes chladnou a vlhkou oblast. Dochází k silnému výparu, který ochlazuje vlivem latentního tepla povrch země. Následně máme tok tepla $H$ negativní zatímco $H_L$ je větší a pozitivní. Může se stát, že i tok tepla v půdě $H_G$ změní znaménko, pokud povrch bude chladnější než půda. Vlhkost půdy má také vliv na albedo povrchu, způsobuje nárůst absorpce slunečního záření, tedy pokles albeda\cite{arya2001}.

\subsection{Vliv vegetace} \label{chap:veg}
\subsubsection{Albedo}
Albedo je koeficient udávající poměr mezi zářením a dopadajícím zářením, nabývá tedy hodnot v intervalu $\langle 0,1\rangle$, kde $0$ znamená, že povrch všechno záření pohlcuje, jde o ideálně černé těleso a $1$ znamená, že vše záření odráží. Už samotný typ povrchu má na albedo významným vliv, kde může dosahovat velmi různých hodnot ať už jde o písek, vodní plochu, odhalenou půdu, vlhkou půdu atd. Růst vegetace a jejího typu má ovšem také významným vliv na albedo s odlišnými hodnotami pro různé typy plodin a různé typy dřevin, ilustrační příklady jsou vidět v tabulce \ref{tab:albedo}. Hodnoty jsou závislé i na úhlu dopadu záření, vliv má i výška porostu nebo roční období\cite{arya2001,alma}.

(+++ tabulka)

\subsubsection{Vliv na tok tepla}
Do členů v rovnici \ref{eq:bilance} můžeme zahrnout i růst vegetace. V tu chvíli musíme počítat s významnou prostorovou závislostí všech toků tepla a záření. Následně jsou pro nás nejdůležitější hodnoty $R_N$, $H$ a $H_L$ nad vegetací, dále $\Delta H_S$ se skládá ze dvou částí a to změna tepla ve vzduchu, vegetaci atp. a změna energie biochemického původu skrze fotosyntézu a přesunu oxidu uhličitého. Latentní teplo se pak skládá z výparu a kondenzace vody a také z transpirace vody listy rostlin, mluvíme pak o evapotranspiraci\cite{arya2001}. 

Vliv na celkový tok energie má i výška porostu, například v lese může být nezanedbatelné uchované teplo v úrovni vegetace, které může způsobit, že prostředí reaguje pomalu na změny teplot a jiných veličin. Pro suchý porost může uchovaná energie dosahovat až $\SI{7}{\%}$ celkového toku energie skrz dopadající záření, situace se ovšem pro vlhký porost obrací a denní uchovaná energie může být záporná\cite{alma}. 

Podle \cite{alma} ovlivňuje teplotu hustota porostu. Jestliže měříme teplotu v řidším lese a sledujeme průběh teploty od země až po koruny stromů tak teplo snadněji prochází vegetací a tudíž budou při zemi podobné teploty jako ve výšce několika metrů nebo až v korunách stromů. Naopak pokud se nacházíme v hustějším lese tak bude teplota růst dříve v oblasti korun stromů a teplota při zemi může mít zpoždění několik hodin a nižší denní maximální hodnoty. V noci také může v řidším lese docházet k inverzi teplot zatímco v hustějším budou teploty podobné napříč porostem. 

Tok tepla v lese je ovlivněn i větrem. Jestliže fouká silný vítr, může docházet k rychlému přenosu energie mimo lesní porost\cite{alma}. 

\subsubsection{Vliv na rychlost větru}
Turbulence vzduchu je zodpovědná za přenos hmoty, tepla a hybnosti v oblasti blízko povrchu země. Bez turbulence by promíchávání probíhalo na molekulární úrovni a řádově menším měřítku. Na míru turbulence a promíchávání má vliv rychlost větru. Vegetace ovlivňuje nejen rychlost větru, ale také její profil. Například koruny stromů můžou rychlost větru výrazně snižovat zatímco, pokud blízko země nejsou menší stromy, keře nebo jiné překážky tak zde může být rychlost větru větší. To má pak vliv na teploty uvnitř porostu, které se mohou více podobat teplotám ve volném vzduchu. Tímto je zřejmé, že vliv na rychlost větru má i přítomnost listů a jejich absence v zimě \cite{alma}. 

\subsubsection{Vliv na vlhkost vzduchu}
V předchozím odstavci jsme zmínili transpiraci flóry, která slouží jako zdroj vodní páry. Ukázková situace v jehličnatém lese může vypadat tak že tlak vodní páry dosahuje dvou maxim a to při povrchu půdy z důvodu nižšího promíchávání vzduchu a výparu a v oblasti korun stromů kvůli listům. Druhé maximum bývá ovšem menší kvůli promíchávání se suchým vzduchem nad korunami stromů. Vzhledem k absolutním rozdíl tlaku vodních par v řádu jednotek milibarů tak změna relativní vlhkost napříč dnem a výškou je dána především teplotním zvrstvením \cite{alma}. 

\subsubsection{Vliv na rosu}
Stromy nejenže stíní povrch země před přímým slunečním zářením a ovlivňují vyzařování dlouhovlnného záření povrchem, ale také ovlivňují množství rosy. Při měření množství ranní rosy v různé vzdálenosti od kmene stromu bylo pozorováno, že množství rosy roste strmě do zhruba $\SI{2}{m}$ od kmene a dále pomaleji, doba po kterou zůstala rosa na zemi také se vzdáleností rostla a od vzdálenosti $\SI{4}{m}$ lehce klesala \cite{alma}.

\subsubsection{Vliv na déšť}
Listnaté, ale i jehličnaté stromy mají vliv na distribuci deště. Největší množství vody se objevuje na vnějším okraji ohraničeném korunou stromu. Stromy bez listů mají na prostorové rozložení srážek pouze minimální vliv. Pod jehličnaté stromy dopadá pouze $\SI{60}{\%}$ až $\SI{90}{\%}$ srážek a v oblasti okraje korun dopadá o $\SI{10}{\%}$ až $\SI{20}{\%}$ více srážek než v místě bez porostu. Lesní porost také při slabém dešti může úplně zabránit dopadu srážek na půdu. Množství vody, které takto dokážou stromy zachytit se může pohybovat od $\SI{1}{mm}$ do $\SI{3}{mm}$. Část vody je také ztracena výparem z povrchu listu, který je podpořen výraznějším promícháváním vzduchu v oblasti horní části korun stromů \cite{alma}.

\subsubsection{Vliv na sníh}
Množství sněhový srážek, které dopadnou na zem v lesním porostu záleží na několika faktorech. Jestliže jde o mokrý sníh pak snadno zůstává v korunách stromů, takto může být v korunách zachyceno až $\SI{10}{cm}$ sněhu. Suchý sníh má naopak snáze dopadne na zem. Množství zachyceného sněhu závisí také na typu vegatace, jestli jde o listnaté nebo jehličnaté stromy, případně jaký druh. Na jaře taje sníh v lese pomaleji a tedy jeho přítomnost je pro místní klima velmi důležité, jelikož funguje jako zásobník vody\cite{alma}.

\subsection{Vliv topografie} \label{chap:topo}
V odstavcích výše jsme vysvětlili jakým způsobem může ovlivnit vegetace podnebí. Ilustrovali jsme, že přítomnost flóry má významný vliv na celou řadu meteorologických proměnných, které následně ovlivňují teplotu v lese a život organismů. Nyní se budeme soustředit na vliv topografie na meteorologické proměnné. Pro standardizované meteorologické stanice je typické, že jsou postavené na posekané travnaté ploše, která není zastíněná a terén není nakloněn, v reálné krajině tohle ovšem neplatí.

\subsubsection{Vliv okraje lesa}
V oblasti okraje lesního porostu máme kontakt rozhraní dvou rozdílných vzduchových hmot. Geografická orientace má vliv na to jak dlouho svítí denní světlo na toto rozhraní. Množství dopadající energie může být pro okraj lesa orientovaný na jih i několikrát větší než pro okraj orientovaný na sever, to platí ovšem např. pro lesy v Evropě. Na rozhranní lesa a otevřené oblasti můžeme dokonce pozorovat vyšší teploty než uvnitř lesa a mimo něj, částečně to může být kvůli sníženému promíchávání vzduchu, větší absorpci slunečního záření, stínění dlouhovlnného záření a nižšímu výparu. Vliv na teplotu má i vítr. Jestliže fouká směrem do lesa, pak je tendence, aby teplota na okraji lesa byla podobná teplotě mimo les a naopak pro vítr směrem z lesa\cite{alma}.

\subsubsection{Vliv sklonu a orientace svahu}
Množství slunečního záření, které dopadne na povrch země závisí na mnoha faktorech. Musíme vzít v potaz zeměpisnou šířku, období v roce, denní čas, sklon svahu a orientaci svahu. Například pro svah s velkým sklonem na severní polokouli, který míří na sever se může stát, že v zimě nedostane po většinu dne žádné sluneční záření. Abychom dostali úplný obrázek musíme započítat i difúzní záření, které je primární složkou záření pokud je zataženo. Difúzní záření není do takové míry ovlivněné sklonem svahu\cite{alma}.

\subsubsection{Vliv údolí}
Nejen sklon a orientace svahu mají vliv na mikroklima. Jeden z dalších důležitých aspektů je jestli se místo, kterým se zabýváme nachází v údolí nebo zda-li jde o kopec. Topografie údolí má vliv hned několika způsoby. Studený vzduch je hustší a tudíž klesá do údolí, kde pak může vznikat kapsa studeného vzduchu, údolí je typické prostředí kde můžeme pozorovat inverzi vzduchu. Údolní topografie může způsobovat to že na dno dopadá menší množství slunečního záření, zároveň je ale dlouhovlnné záření částečně stíněno okraji údolí. Do údolí hůře proniká vítr a tudíž je zde snížená turbulence a tím i turbulencí předávané teplo. Pro velmi úzké a hluboké propadliny je půda významným zdrojem tepla. Všechny tyto faktory mají opačný vliv v případě kopců a obecně vyvýšených míst\cite{alma}. 

\subsubsection{Vliv topografie na vítr}
Vítr ovlivněný topografií terénu můžeme rozdělit na tři typy podle původu: kompenzační vítr, horský a údolní vítr a vítr spojený se sklonem svahu. V prvním případě jde o vítr způsobený nevyváženým ohřevem zemského povrchu. Na teplejším místě se izobary začnou zvedat a tím vzniká horizontální gradient tlaku mezi chladnou a teplejší oblastí. Vítr spojený se sklonem svahu má v noci tendenci jít z vyšších nadmořských poloh dolů a naopak během dne. Můžeme mít také situaci, kdy pozorujeme údolí, které se zároveň svažuje. Zde už je situace složitější.

Při svítání se začíná povrch údolí ohřívat. Vzduch u strany údolí stoupá nahoru, ale údolní vítr stále ještě fouká směrem dolů. Během dne údolní vítr oslabuje až dojde k obrácení jeho směru a údolím fouká nahoru. Později odpoledne vítr po svazích začne ustávat a fouká pouze údolní vítr do vyšších nadmořských výšek. Následně chladnější vzduch ze stran údolí začne klesat a v noci nakonec údolní vítr začne opět vát dolů stejně jako vítr po svazích údolí. Toto popisuje ukázkovou situaci za letního dne bez oblačnosti.

Vítr ovlivněný topografií může mít další podoby, které ovšem pro tuto práci nejsou zásadní. Můžeme zmínit mořský vánek mezi střídavě rychle se ohřívajícím pobřežím a relativně studeném moři ve dne a rychle chladnoucím pobřeží a relativně teplému moři v noci. Významný vítr spojený s horskou topografií pak dostává různé názvy jako fén v Alpách nebo mistrál ve Francii\cite{alma}.

\subsection{Rozdíl mezi teplotou při povrchu země a ve standardní výšce}
Teplotou ve standardní výšce se typicky myslí ve výšce $\SI{1}{m}-\SI{2}{m}$. Jestliže máme situaci, kdy na povrch svítí přímé sluneční záření, tak se můžou vyskytovat výrazné gradienty teplot a to až $\SI{10}{K/mm}-\SI{20}{K/mm}$. Tyto výrazné gradienty jsou ovliněné mnoha faktory jak například druh povrchu nebo jeho vlhkostí a dalšími jak bylo diskutováno výše. Dále kvůli nezanedbatelné velikosti teplotních senzorů je netriviální měřit teplotu povrchu\cite{arya2001}. 

\section{Analýza faktorů ovlivňující teplotu vzduchu v lesním porostu}
V předchozích odstavcích jsme diskutovali vliv vegetace a topografie na různé meteorologické proměnné. Tuto souvislost v následující části budeme diskutovat na konkrétní studii týkající se výzkumu ploch po střední Evropě.

\subsection{Vliv topografie a struktury krajiny na teplotu}
Při hledání vlivu topografie na rozdíl mezi teplotou v lesním porostu a na nejbližší meteorologické stanici ve střední Evropě se ve výzkumu \cite{ZellwegerFlorian2019Sdou} hledala souvislost pomocí následujících prediktorů.

\begin{itemize}
	\item Plocha pokrytá lesním porostem v okruhu $\SI{250}{m}$ vyjádřená v procentech. Zde nebyla nalezena spojitost s rozdílem teplot.
	\item Vzdálenost k okraji lesa byla nevýznamným prediktorem. 
	\item Vzdálenost k nejbližšímu pobřeží a výška nad mořem byly nejsilnějšími prediktory, zároveň jde ovšem o silně korelované veličiny. Slabšími prediktory pak bylo stočení svahu k severu/jihu, sklon svahu a hodnota udávající zda-li jde spíše o údolí nebo vyvýšenné místo.
\end{itemize}

Tyto výsledky ovšem nejsou nutně konzistentní mezi různými studiemi. Například podle \cite{GreiserCaroline2018Mmmi} je plocha pokrytá lesním porostem důležitou hodnotou a může vést ke zvýšení minimální denní teploty až o $\SI{3}{\degree C}$ nebo vzdálenost k okraji lesa je středně silný prediktor.

\subsection{Vliv porostu na teplotu}
Druhou skupinou prediktorů kterou se \cite{ZellwegerFlorian2019Sdou} zabýval byly proměnné týkající se porostu v blízkém okolí čidla.

\begin{itemize}
	\item Zápoj má nelineární a záporný vliv na maximální teploty. Pod hodnotou $\SI{89}{\%}$ dochází k prudkému nárůstu vlivu vegetace.
	\item Otevřenost porostu vyjádřená jako část viditelné oblohy a plocha koruny stromů také má záporný nelineární vztah k maximální teplotě.
	\item Procento plochy pokryté dřevinami nad určitý průměr je pouze slabým prediktorem. 
	\item Výška stromu na kterém bylo čidlo upevněno naopak souvisí s rozdílem maximální teplot kladně, ale opět zde vztah není příliš silný.
	\item Schopnost porostu vytvářet stín podle typu dřeviny je středním prediktorem rozdílu maximálních teplot.
\end{itemize}

\subsection{Vliv meteorologických podmínek na teplotu}
Jak bylo zmíněno v předchozí části tak na rozdíl mezi teplotou mimo porost ať už lesní nebo jiný má vliv mnoho faktorů od topografických po konkrétní typ porostu. Tyto faktory se dají vyjádřit mnoha různými prediktory jejichž vliv můžeme sledovat. Aktuální stav počasí je ovšem další faktorem, který ovlivňuje naměřené teploty. Tím může být oblačnost, vlhkost, sněhová pokrývka, srážky, rychlost větru a roční doba.

V praktické části práce se budeme zabývat tím jaký vliv mají tyto proměnné na rozdíl mezi teplotou naměřenou v lesním porostu v národním parku Šumava ve výšce $\SI{2}{m}$ a při povrchu, tedy $\SI{0}{cm}$ a $\SI{15}{cm}$. Použitá čidla kromě teploty ještě půdní vlhkost, ostatní proměnné jsou z nejbližší meteorologické stanice.

\section{Popis měření meteorologický veličin} \label{chap:measure}
\subsection{Měření na meteorologických stanicích}
Teplota vzduchu by podle Světové meteorologické organizace (WMO) měla být měřena ve výšce $\SI{1.25}{m}$ až $\SI{2}{m}$\cite{wmoGuidance2021}. V České republice Český hydrometeorologický ústav provozuje celou řadu meteorologických stanic různého charakteru, ať už jde o SYNOP stanice nebo jiné +++. Měření probíhá standardizovaným způsobem, stanice by měla být na trávníku nikoliv například na asfaltu a měření probíhají v pravidelných intervalech. Pro synoptické stanice je to například v takzvaných hlavních termínech ($00$, $06$, $12$ a $18$ UTC) a případně vedlejších ($03$, $09$, $15$, $21$ hodin UTC). Podobně jsou na stanicích měřeny i další veličiny jako například, vlhkost, výška sněhu, část zatažené oblohy, tlak a další. 

\subsection{Měření v terénu pomocí čidel}
Pomocí meteorologických stanic není možné získat detailní informace o teplotě a dalších veličinách na malých prostorových škálách. Standartní meteorologické stanice nemůžou změřit teplotu například v lesním porostu. Data použitá v této práci jsou měřená pomocí TMS (Temperature-Moisture-Sensor) logger, kromě čidel jako například Thermochron iButtons, je možné využít radiometrie (+++) a z vyzářeného záření z povrchu pomocí Stefan-Boltzmannova zákona spočítat teplotu povrchu. Poslední metoda má ovšem své omezení pokud chceme znát teplotu v lesním porostu. 

\subsubsection{TMS4 a T1 logger} \label{chap:loggers}
Čidlo TMS4 logger používané pro sběr dat vědci a vědkyněmi z Botanického ústavu Akademie věd České republiky je konstruováno tak, aby měřilo podmínky, které prožívá malá bylinna. Je tedy vysoké $\SI{15}{cm}$ nad povrch země a sahá do hloubky $\SI{14}{cm}$. Je opatřeno třemi teplotními senzory ve výškách $\SI{15}{cm},\ \SI{0}{cm},\ \SI{-8}{cm}$. Až do hloubky $\SI{14}{cm}$ je měřena volumetrická půdní vlhkost. Vršek data loggeru je opatřen optickým stíněním z bílého plastu chránící horní senzor před přímým slunečním zářením a stejně tak je odstíněn i senzor při povrchu země. 

Teplotní senzor měří s přesností $\SI{\pm 0.5}{\degree C}$ a funguje na intervalu $\SI{-55}{\degree C}$ až $\SI{125}{\degree C}$. Měření volumetrické půdní vlhkostí je založené na time-domain-transmission (TDT), kdy jsou skrze obvod vysílány elektromagnetické pulzy a množství detekovaných pulzů je přímo úměrné vlhkosti. 

Při srovnání TMS4 loggeru se standartní meteorologickou stanicí typu METEOS 5 byla pozorována větší variabilita u teplot v $\SI{15}{cm}$ než ve $\SI{2}{m}$. TMS4 loggeru byl instalován na krátkém posekaném trávníku, tři metry od stanice a rozdíl teplot při hodinovém měření se pohyboval od $\SI{+8.45}{\degree C}$ do $\SI{-6.05}{\degree C}$. Průměrné denní teploty naměřené z TMS4 loggeru byly systematicky nižší, průměrný rozdíl ovšem pouze $\SI{-0.58}{\degree C}$, v zimě jsou tyto rozdíly větší a to $\SI{-2.02}{\degree C}$ až $\SI{0.6}{\degree C}$. Opačný trend pak můžeme pozorovat, pokud není teplotní senzor v $\SI{15}{cm}$ odstíněn plastovým krytem. V létě byly teploty nižší až o $\SI{5.08}{\degree C}$, ale v zimě byly podobné. Teploty z půdních senzorů jak TMS4 loggeru tak stanice METEOS 5 se lišily velmi málo, rozdíl mezi naměřenými teplotami byl tedy zřejmě způsoben primárně polohou senzorů a použitým stíněním než jejich typem. Rozdíly mězi naměřenými teplotami nad zemí byly ještě výraznější pro proměnné, které jsou často používány v ekologických studiích jako například kvantily extrémních teplot\cite{WildJan2019Caer}. 

Data měřená při povrchu země jsou na některých místech doplněna stejnými teplotními senzory ve výšce $\SI{2}{m}$ upevněnými na kmenu stromu a také opatřena plastovým stínítkem.

\section{Národní park Šumava} \label{chap:sumavabavorskyles}
Čidla TMS logger z kterých pocházejí data použitá v této práci jsou v oblasti Národní parku Šumava. Národní park Šumava vyhlášený v roce 1991 má rozlohu $\SI{680}{km^2}$. Jeho výška se pohybuje od $\SI{600}{m.\ n.\ m.}$ do $\SI{1378}{m.\ n.\ m.}$. Nejvyšší bod je vrchol Plechý. Více než $\SI{80}{\%}$ územi pokrývá les. Šumava je v rámci Evropy jedno z nejstarších pohoří, tvořené je metamorfovanými horninami jako ruly a svory a žulovými masivy. Typické pro toto území jsou horské smrčiny, rašeliniště, karová jezera a pralesní porosty.

