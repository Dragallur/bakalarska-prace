\chapter{Analýza dat} \label{chap:analysis}
V první části druhé kapitoly se zaměříme na popis struktury použitých dat. V druhé části analyzujeme autokorelační a prostorovou složku dat, a hledat vhodný model pro vysvětlení rozdílů naměřených teplot.

\section{Využitá data z ČHMÚ}
Pro hledání souvislosti mezi meteorologickými podmínkami a rozdílem teplot na čidlech poblíž země jsme využili data z meteorologických stanic Churáňov, Borová Lada, Kvilda, Horská Kvilda a Javoří pila. Nejvíce informací jsme získali ze synoptické stanice Churáňov. Z dostupných dat jsme pro další analýzu využili data o aktuální teplotě ve výšce $\SI{2}{m}$, rychlosti větru, výšce sněhové pokrývky, hodinových srážek a oblačnosti. Z ostatních stanic jsme měli k dispozici hodinová data o výšce sněhové pokrývky a pro Borovou Ladu informace o úhrnu srážek.

Analýzu provádíme pro období od 12.10.2019 do 21.5.2021 i když v případě některých čidel kvůli výpadkům měření bude zpracovávaný interval kratší (viz kapitola \ref{chap:data_buav}).

Na obrázku \ref{fig:chmuukazka} můžeme vidět ukázku dat z meteorologické stanice Churáňov. Maximální teplota ($\SI{27}{\degree C}$) za období 12.10.2019 až 20.5.2021 byla naměřena 21.8.2020 a minimální ($\SI{-16.1}{\degree C}$) byla naměřena 11.2.2021 a 13.2.2021. Největší sněhová pokrývka byla $\SI{48}{cm}$ 31.1.2021.

\begin{figure}
	\centering
	\begin{subfigure}{0.45\textwidth}
  \includegraphics[width=\textwidth]{img/ch2/hist_temp_synop_bymonth.png}
		\caption{Okamžité denní teploty ve výšce $\SI{2}{m}$ nad zemí rozdělené podle měsíců v roce.}
		\label{fig:synop_temperature}
	\end{subfigure}
	\hfill
	\begin{subfigure}{0.45\textwidth}
  \includegraphics[width=\textwidth]{img/ch2/hist_snow_synop_bymonth.png}
		\caption{Denní sněhová pokrývka.}
		\label{fig:synop_snowcm}
	\end{subfigure}
	\hfill
	\begin{subfigure}{0.45\textwidth}
  \includegraphics[width=\textwidth]{img/ch2/hist_prec_synop_bymonth.png}
		\caption{Denní suma srážek.}
		\label{fig:synop_prec}
	\end{subfigure}
	\hfill
	\begin{subfigure}{0.45\textwidth}
  \includegraphics[width=\textwidth]{img/ch2/hist_wind_synop_bymonth.png}
		\caption{Histogram pozorovaných okamžitých rychlostí větru.}
		\label{fig:synop_ffkmh}
	\end{subfigure}
	\caption{Data ze synoptické stanice Churáňov za období 12.10.2019 až 20.5.2021. Rychlost větru zobrazujeme jako obyčejný histogram, protože jde o informativnější graf než dělení podle měsíců v roce. Horizontální čára označuje medián, hodnoty uvnitř krabičky jsou mezi 25 ($Q_1$) a 75 percentilem ($Q_3$) a odlehlé body jsou ty, které mají hodnotu větší nebo menší o 1.5násobek velikosti krabičky $Q_3-Q_1$.}
	\label{fig:chmuukazka}
\end{figure}

Meteorologická stanice Churáňov měřící podle principů popsaných v kapitole \ref{chap:meteostations}. Stanice zaznamenává oblačnost každou hodinu od 6:00 do 20:00 UTC. Maximální teploty jsou typicky dosažené během dne, kdy většinou oblačnost známe. Například v zimě pod sněhem může ale maximální denní teplota být dosažena i v noci. Minimální denní teploty na druhou stranu typicky nastávají během ranních hodin, po východu Slunce, ale ojediněle i před ním (viz kapitola \ref{chap:showingoffdata} a obrázek \ref{fig:hours}). Pro minimální teploty nám tedy často chybí údaj o oblačnosti. Vyřazení těchto hodnot by vložilo velké zkreslení do dat, tudíž jsme se rozhodli využít data z projektu ERA5.

Pro každou hodnotu, která chybí ze staničního měření použijeme nejbližší hodnotu z ERA5 a to konkrétně z $\SI{49}{\degree}$ severní šířky a $\SI{13.5}{\degree}$ východní délky. Tato data byla vzdálená od stanice Churáňov $\SI{14.8}{km}$. Tímto zavádíme do dat určitou nejistotu, ať už kvůli prostorovému skoku, ale také zkreslení, když používáme data z odlišných zdrojů, kdy jedním jsou měřená a druhým pouze modelována. Pro robustnější modely a interpretaci by bylo vhodné spočítat modely pouze s daty z ERA5 a porovnat výsledky. Pro sofistikovanější přístup bychom mohli také využít data z více bodů a provést prostorovou interpolaci na místa čidel. Toto jde ovšem za rámec této práce.

\section{Data z Botanického ústavu Akademie věd}\label{chap:data_buav}
Data poskytnutá Botanickým ústavem Akademie věd České republiky byla naměřená dvěma typy čidel popsanými v kapitole \ref{chap:loggers}. Nadále se budeme zabývat pouze těmi plochami, které jsou opatřeny jak pozemními čidly, tak čidly ve standartní výšce $\SI{2}{m}$. Na obrázku \ref{fig:rozlozenicidel} můžeme vidět jejich prostorové rozložení, celkově jde o 157 čidel. 

Umístění čidel bylo vybíráno tak, aby pokryly gradienty nadmořské výšky (5 tříd), potenciální solární radiace, určující množství záření, které dopadá na zem (3 třídy) a topografického vlhkostního indexu, který kombinuje svažitost a akumulaci vody v terénu (3 třídy). Plochy byly dále doplněny tak, aby bylo rovnoměrně pokryto území národních parků, a aby se nevyskytovaly poblíž turistických stezek.

\begin{figure}
	\centering
	\includegraphics[width=0.95\textwidth]{img/rozlozenicidel.pdf}
	\caption{Rozložení čidel a meteorologický stanic v Národním parku Šumava ($N=112$) a Národním parku Bavorský les ($N=45$)}
	\label{fig:rozlozenicidel}
\end{figure}

Data vykazují malou chybovost, duplicitní a chybějící záznamy byly vyřazeny. Dále byla data vizuálně překontrolována, jestli neobsahují očividně chybná data. Části, kdy byla čidla např. povytažená ze země (pozná se podle hodnot půdní vlhkosti), byly nahrazeny hodnotami NA. Podobně pokud čidlo T1 spadlo ze stromu, tak jsou hodnoty nahrazeny NA. Toto čištění dat provedli RNDr. Josef Brůna, Ph.D., doc. Ing. Jan Wild, Ph.D. a další, s jejichž svolením jsou data využitá v této práci. Ve velmi ojedinělých případech chyběly odpovídající hodnoty teplot ve $\SI{2}{m}$, v době, kdy při zemi nastalo denní maximum nebo minimum. Pokud existovaly hodnoty až 30 minut starší, tak jsme použili tyto, v opačném případě jsme čidlo pro daný den vyřadili, šlo o jednotky případů.

Dostupnost dat z čidel je vidět na obrázku \ref{fig:dostupnostdat}, vidíme zde dvě skupiny čidel. Čidla nacházející se v Národním parku Bavorský les mají dostupná data pro cca 400 dnů. Čidla z Národního parku Šumava mají dostupná data pro téměř 600 dnů. Na obrázku \ref{fig:dostupnostdnu} vidíme, na kolika čidlech jsou zastoupeny jednotlivé dny.

\begin{figure}
	\centering
	\begin{subfigure}{0.45\textwidth}
  \includegraphics[width=\textwidth]{img/hist_numofdayavailability.png}
	\caption{Histogram ukazující množství dostupných dnů pro jednotlivá čidla}
	\label{fig:dostupnostdat}
	\end{subfigure}
	\hfill
	\begin{subfigure}{0.45\textwidth}
  \includegraphics[width=\textwidth]{img/date_availability.png}
	\caption{Graf zastoupení čidel pro jednotlivé dny}
	\label{fig:dostupnostdnu}
	\end{subfigure}
	\caption{Dostupnost dat z čidel}
\end{figure}

Pro samotné zpracování jsme vyřadili poslední 4 dny v květnu, kvůli nižšímu počtu aktivních čidel. Tímto se dostáváme na interval dat od 12.10.2019 do 17.5.2021.

\section{Insolace}
Na obrázku \ref{fig:insolacelogger} můžeme vidět hodnoty insolace spočtené podle odstavce \ref{chap:insolation} pro čidlo, které je nejblíž stanici Churáňov. Sinusoida nám téměř určuje hodnotu maximální denní insolace. Maximum insolace většinou nastává dříve než maximální teploty. Dále také pracujeme na časovém měřítku 15 minut a je tedy malá pravděpodobnost, že nastane maximální denní teplota ve stejnou dobu jako maximální denní insolace. Nulové hodnoty odpovídají tomu, že denní maximální teplota nastala v noci, to způsobuje přítomnost sněhu v zimě.

\begin{figure}
	\centering
	\includegraphics[width=0.65\textwidth]{img/ch2/insolation_max15cmNPS_4311_D_TMS.png}
	\caption{Hodnoty insolace na čidlu nejblíže stanici Churáňov v době dosažení maximální denní teploty ve výšce $\SI{15}{cm}$.}
	\label{fig:insolacelogger}
\end{figure}

\section{Ukázka použitých dat}\label{chap:showingoffdata}
Dále se podíváme na ukázku dat naměřených na čidlech. Na obrázcích \ref{fig:hours} můžeme vidět, kdy nastávaly maximální a minimální teploty ve výškách $\SI{0}{cm}$ a $\SI{15}{cm}$ nad zemí, denní hodina je uvedená v UTC, nikoliv SEČ nebo SELČ. U maximálních teplot si můžeme všimnout kromě maxima v době kolem 10 UTC také menšího maxima a odlehlých hodnot způsobených přítomností sněhu v zimě. Podobně měl sníh vliv i na dobu minimálních teplot v zimě.

\begin{figure}
	\centering
	\begin{subfigure}{0.45\textwidth}
  \includegraphics[width=\textwidth]{img/hist_hourmax15cm.png}
		\caption{}
		\label{fig:hourmax15cm}
	\end{subfigure}
	\hfill
	\begin{subfigure}{0.45\textwidth}
  \includegraphics[width=\textwidth]{img/hist_hourmax0cm.png}
		\caption{}
		\label{fig:hourmax0cm}
	\end{subfigure}
	\hfill
	\begin{subfigure}{0.45\textwidth}
  \includegraphics[width=\textwidth]{img/hist_hourmin15cm.png}
		\caption{}
		\label{fig:hourmin15cm}
	\end{subfigure}
	\hfill
	\begin{subfigure}{0.45\textwidth}
  \includegraphics[width=\textwidth]{img/hist_hourmin0cm.png}
		\caption{}
		\label{fig:hourmin0cm}
	\end{subfigure}
	\caption{Denní doba (UTC) dosažení maximální resp. minimální teploty v $\SI{15}{cm}$ resp. v $\SI{0}{cm}$ nad zemí na čidle nejblíže stanici Churáňov}
	\label{fig:hours}
\end{figure}

Pro ilustraci se podívejme na konkrétní hodnoty teplot pozorované na nejbližším čidlu meteorologické stanice Churáňov. Na obrázcích \ref{fig:maxtemp} můžeme vidět průběh denních minim a maxim na jednom z čidel, celkově jde o 587 dní, období od 12.10.2019 do 20.5.2021. Na obrázcích \ref{fig:2mhours} můžeme vidět hodnoty teplot naměřených ve výšce $\SI{2}{m}$ na čidle zavěšeném na stromě poblíž pozemním čidlům, jde o hodnoty naměřené v době denního teplotního maxima a minima ve výšce $\SI{15}{cm}$.

\begin{figure}
	\centering
	\begin{subfigure}{0.45\textwidth}
  \includegraphics[width=\textwidth]{img/maxtempmax15cm.png}
		\caption{}
		\label{fig:maxtempmax15cm}
	\end{subfigure}
	\hfill
	\begin{subfigure}{0.45\textwidth}
  \includegraphics[width=\textwidth]{img/maxtempmax0cm.png}
		\caption{}
		\label{fig:maxtempmax0cm}
	\end{subfigure}
	\hfill
	\begin{subfigure}{0.45\textwidth}
  \includegraphics[width=\textwidth]{img/maxtempmin15cm.png}
		\caption{}
		\label{fig:maxtempmin15cm}
	\end{subfigure}
	\hfill
	\begin{subfigure}{0.45\textwidth}
  \includegraphics[width=\textwidth]{img/maxtempmin0cm.png}
		\caption{}
		\label{fig:maxtempmin0cm}
	\end{subfigure}
	\hfill
	\begin{subfigure}{0.45\textwidth}
  \includegraphics[width=\textwidth]{img/ch2/rollmean.png}
		\caption{}
		\label{fig:rollmean}
	\end{subfigure}
	\caption{Průběh denních maximální resp. minimálních teplot ve výšce $\SI{15}{cm}$ resp. $\SI{0}{cm}$ nad zemí na čidle nejblíže stanici Churáňov (a-d). Klouzavý týdenní průměr (e) pro grafy (a-d) s barvami, kde (a) červená, (b) modrá, (c) zelená a (d) černá.}
	\label{fig:maxtemp}
\end{figure}

\begin{figure}
	\centering
	\begin{subfigure}{0.45\textwidth}
  \includegraphics[width=\textwidth]{img/2mmaxtempmax15cm.png}
		\caption{Teploty naměřené ve stejnou dobu jako maximální teploty v $\SI{15}{cm}$}
		\label{fig:2mmaxtempmax15cm}
	\end{subfigure}
	\hfill
	\begin{subfigure}{0.45\textwidth}
  \includegraphics[width=\textwidth]{img/2mmaxtempmin15cm.png}
		\caption{Teploty naměřené ve stejnou dobu jako minimální teploty v $\SI{15}{cm}$}
		\label{fig:2mmaxtempmin15cm}
	\end{subfigure}
	\caption{Teploty ve výšce $\SI{2}{m}$ nad zemí na čidle nejblíže stanici Churáňov v době kdy nastalo maximum resp. minimum na čidle ve výšce $\SI{15}{cm}$}
	\label{fig:2mhours}
\end{figure}

Na obrázku \ref{fig:hist_diff} můžeme vidět histogramy rozdílů teplot pro jednotlivé výšky a pro maxima a minima.

\begin{figure}
	\centering
	\begin{subfigure}{0.45\textwidth}
  \includegraphics[width=\textwidth]{img/ch2/hist_diff_max15cm.png}
		\caption{Rozdíl maximálních teplot v $\SI{15}{cm}$ a $\SI{2}{m}$. $\text{M} = 0.69$, $\text{MD} = 0.25$.}
		\label{fig:hist_diff_max15cm}
	\end{subfigure}
	\hfill
	\begin{subfigure}{0.45\textwidth}
  \includegraphics[width=\textwidth]{img/ch2/hist_diff_max0cm.png}
		\caption{Rozdíl maximálních teplot v $\SI{0}{cm}$ a $\SI{2}{m}$. $\text{M} = -0.24$, $\text{MD} = -0.25$.}
		\label{fig:hist_diff_max0cm}
	\end{subfigure}
	\hfill
	\begin{subfigure}{0.45\textwidth}
  \includegraphics[width=\textwidth]{img/ch2/hist_diff_min15cm.png}
		\caption{Rozdíl minimálních teplot v $\SI{15}{cm}$ a $\SI{2}{m}$. $\text{M} = -0.36$, $\text{MD} = -0.125$.}
		\label{fig:hist_diff_min15cm}
	\end{subfigure}
	\hfill
	\begin{subfigure}{0.45\textwidth}
  \includegraphics[width=\textwidth]{img/ch2/hist_diff_min0cm.png}
		\caption{Rozdíl miniálních teplot v $\SI{0}{cm}$ a $\SI{2}{m}$. $\text{M} = 1.14$, $\text{MD} = 0.9375$.}
		\label{fig:hist_diff_min0cm}
	\end{subfigure}
	\caption{Histogramy rozdílů maximální, resp. minimální teploty v $\SI{15}{cm}$, resp. v $\SI{0}{cm}$ a ve $\SI{2}{m}$. Ke každému histogramu uvádíme odpovídající hodnotu průměru $\text{M}$ a mediánu $\text{MD}$.}
	\label{fig:hist_diff}
\end{figure}


\section{Metody analýzy dat}\label{chap:methods}
Cílem následující části je ukázat jakým způsobem byla data zpracována, proč byl vybrán daný model a ukázat metody ověření předpokladů.

\subsection{Korelace dat}
Z povahy naměřených dat je zřejmé, že zde existuje časová autokorelace mezi naměřenými maximálními nebo minimálními teplotami. Autokorelace se pak projevuje i u rozdílu teploty naměřené blízko země a ve $\SI{2}{m}$. Na obrázku \ref{fig:acf} vidíme autokorelační funkci pro jedno z čidel.

\begin{figure}
	\centering
	\includegraphics[width=0.55\textwidth]{img/ch2/acfNPS_4311_D_TMS.png}
	\caption{Časová autokorelační funkce pro rozdíl teplot mezi maximální teplotou v $\SI{15}{cm}$ a teplotou ve $\SI{2}{m}$ na páru čidel nejblíže meteorologické stanici Churáňov.}
	\label{fig:acf}
\end{figure}

Autokorelace není takto významná pro všechna čidla, ale i tak nám vylučuje možnost využít jednoduché mnohonásobné lineární regrese.

Kvůli prostorové struktuře dat musíme otestovat prostorovou korelaci. Využijeme teorii popsanou v kapitole \ref{chap:variogram}. Na obrázcích \ref{fig:variograms} vidíme dvanáct semivariogramů pro každý první den měsíce v roce 2020. Kvůli lišícím se hodnotám semivariance nemůžeme semivariogramy zakreslit do jednoho grafu. Ze semivariogramů můžeme usoudit, že v datech neexistuje významná prostorová korelace, kterou bychom museli v modelu zohlednit (prostorovou korelaci, bychom očekávali hlavně v prvních deseti kilometrech viz kapitola \ref{chap:variogram}).

\begin{figure}
	\centering
	\begin{subfigure}{0.30\textwidth}
		\includegraphics[width=\textwidth]{img/ch2/variograms/variogram_max15cm1.png}
		\caption{}
		\label{fig:variogram1}
	\end{subfigure}
	\hfill
	\begin{subfigure}{0.30\textwidth}
		\includegraphics[width=\textwidth]{img/ch2/variograms/variogram_max15cm2.png}
		\caption{}
		\label{fig:variogram2}
	\end{subfigure}
	\hfill
	\begin{subfigure}{0.30\textwidth}
		\includegraphics[width=\textwidth]{img/ch2/variograms/variogram_max15cm3.png}
		\caption{}
		\label{fig:variogram3}
	\end{subfigure}
	\hfill
	\begin{subfigure}{0.30\textwidth}
		\includegraphics[width=\textwidth]{img/ch2/variograms/variogram_max15cm4.png}
		\caption{}
		\label{fig:variogram4}
	\end{subfigure}
	\hfill
	\begin{subfigure}{0.30\textwidth}
		\includegraphics[width=\textwidth]{img/ch2/variograms/variogram_max15cm5.png}
		\caption{}
		\label{fig:variogram5}
	\end{subfigure}
	\hfill
	\begin{subfigure}{0.30\textwidth}
		\includegraphics[width=\textwidth]{img/ch2/variograms/variogram_max15cm6.png}
		\caption{}
		\label{fig:variogram6}
	\end{subfigure}
\hfill
	\begin{subfigure}{0.30\textwidth}
		\includegraphics[width=\textwidth]{img/ch2/variograms/variogram_max15cm7.png}
		\caption{}
		\label{fig:variogram7}
	\end{subfigure}
\hfill
	\begin{subfigure}{0.30\textwidth}
		\includegraphics[width=\textwidth]{img/ch2/variograms/variogram_max15cm8.png}
		\caption{}
		\label{fig:variogram8}
	\end{subfigure}
\hfill
	\begin{subfigure}{0.30\textwidth}
		\includegraphics[width=\textwidth]{img/ch2/variograms/variogram_max15cm9.png}
		\caption{}
		\label{fig:variogram9}
	\end{subfigure}
\hfill
	\begin{subfigure}{0.30\textwidth}
		\includegraphics[width=\textwidth]{img/ch2/variograms/variogram_max15cm10.png}
		\caption{}
		\label{fig:variogram10}
	\end{subfigure}
\hfill
	\begin{subfigure}{0.30\textwidth}
		\includegraphics[width=\textwidth]{img/ch2/variograms/variogram_max15cm11.png}
		\caption{}
		\label{fig:variogram11}
	\end{subfigure}
\hfill
	\begin{subfigure}{0.30\textwidth}
		\includegraphics[width=\textwidth]{img/ch2/variograms/variogram_max15cm12.png}
		\caption{}
		\label{fig:variogram12}
	\end{subfigure}
	\caption{Semivariogramy pro jednotlivé první dny měsíců v roce 2020.}
	\label{fig:variograms}
\end{figure}

\subsection{Lineární model se smíšenými efekty}
Pro analýzu vlivu meteorologických proměnných je vhodný lineární model se smíšenými efekty popsaný v kapitole \ref{chap:lme}. Jako náhodný efekt, jehož reálná hodnota pro nás není důležitá, určíme identitu (název) páru pozemního čidla a čidla ve výšce $\SI{2}{m}$. Vybereme identitu čidla jako náhodný efekt, protože nás ve výsledném modelu nezajímají koefienty každého čidla, nejde pro nás o důležitý prediktor. Také předpokládáme, že díky tomu, že jsou čidla rovnoměrně rozložená po ploše národních parků tak jde o výběr s normálním rozdělením z možných pozic čidel. Když vyřadíme měření, pro která chybí některá data z meteorologických stanic, a poslední čtyři dny měření, kdy máme data z menšího množství čidel, tak máme $76627$ měření, hodnoty pro jiné kombinace čidel a maximálních/minimálních teplot jsou v tabulce \ref{tab:seznammodelu}.

Zpracovávaná data mají šest prediktorů: insolace, srážky za poslední hodinu, celková sněhová pokrývka, oblačnost, rychlost větru a půdní vlhkost. Oblačnost nabývá diskrétních hodnot, je vyjádřena v osminách celkové oblohy. Data z reanalýzy ERA5 jsou vyjádřena v procentech, všechny hodnoty tedy převedeme do intervalu $0$ až $1$. Rychlost větru je měřená v diskrétních násobcích $\SI{1}{km/h}$. Ostatní prediktory, stejně jako rozdíl mezi teplotami při povrchu země, který se snažíme vysvětlit, jsou spojité proměnné. 

Pro výpočet lineárního modelu se smíšenými efekty použijeme funkci \texttt{lme} z balíčku \texttt{nlme} programovacího jazyka \texttt{R}.

Pro každý model budeme ověřovat předpoklady lineární modelu se smíšenými efekty. Krátce ilustrujeme problémy na jednom modelu, s kterými se budeme v kapitole \ref{chap:ch3} potýkat. Pracujme tedy s maximální denní teplotou ve výšce $\SI{15}{cm}$ a jejím rozdílem od teploty naměřené ve $\SI{2}{m}$. Porovnáme mezi sebou residuály transformované proměnné, kterou se snažíme vysvětlit, a to bez transformace, s logaritmem, odmocninou a třetí odmocninou. Zároveň data mají silnou autokorelaci a tudíž, použijeme pro autokorelační strukturu model ARMA s parametry $p=2$ a $q=1$ (tyto hodnoty byly vybrány vyzkoušením několika kombinací). Na obrázcích \ref{fig:qq} vidíme kvantil-kvantilový graf pro jednotlivé transformace. Hodnoty rozdílu teplot jsou ovšem i záporné, tudíž místo jednoduchého logaritmu použijeme transformaci \eqref{eq:logtrans}. Podobně ošetříme záporné hodnoty i pro ostatní transformace, jako například 3. odmocninu \eqref{eq:curttrans}. V datech se objevují i hodnoty rozdílu teplot $\Delta t=0$ pro které nemůžeme spočítat logaritmus. Jde ovšem o hodnoty pouze blízké nule, způsobené konečnou přesností čidel, tudíž na tyto hodnoty aplikujeme funkci \texttt{jitter}. Tato funkce přidá k hodnotám šum a používáme defaultní parametr $\text{factor}=0$. Tímto je odlišíme od nuly a následně provedeme transformaci.

\begin{gather}
	T \mapsto \mathrm{sign}(T)\cdot \ln\left|T\right| \label{eq:logtrans}\\
	T \mapsto \mathrm{sign}(T)\cdot \sqrt[3]{\left|T\right|} \label{eq:curttrans}
\end{gather}

\begin{figure}
	\centering
	\begin{subfigure}{0.45\textwidth}
  \includegraphics[width=\textwidth]{img/ch2/qq_modmax15cm_none.png}
		\caption{Bez transformace}
		\label{fig:qq_none}
	\end{subfigure}
	\hfill
	\begin{subfigure}{0.45\textwidth}
  \includegraphics[width=\textwidth]{img/ch2/qq_modmax15cm_log.png}
		\caption{Přirozený logaritmus}
		\label{fig:qq_log}
	\end{subfigure}
	\hfill
	\begin{subfigure}{0.45\textwidth}
  \includegraphics[width=\textwidth]{img/ch2/qq_modmax15cm_sqrt.png}
		\caption{Druhá odmocnina}
		\label{fig:qq_sqrt}
	\end{subfigure}
	\hfill
	\begin{subfigure}{0.45\textwidth}
  \includegraphics[width=\textwidth]{img/ch2/qq_modmax15cm_curt.png}
		\caption{Třetí odmocnina}
		\label{fig:qq_curt}
	\end{subfigure}
	\caption{Kvantil-kvantilový graf pro jednotlivé transformace vysvětlované proměnné.}
	\label{fig:qq}
\end{figure}

Vidíme, že nejlépe normálnímu rozdělení odpovídá transformace pomocí třetí odmocniny, a proto s ní budeme nadále pracovat. Transformace ovšem není dokonalá, stále jsme se úplně nezbavili šikmosti (skewness). Formální testy normálity rozdělení jako například Shapiro-Wilkův test se zde nehodí, neboť jsou velmi náchylné na malé odchylky od normálního rozdělení \parencite{shapirowilk}.

Homoskedasticitu budeme ověřovat graficky, pomocí srovnání fitovaných hodnot a residuálů modelu. Pokud nebyl porušen předpoklad homoskedasticity, tak nesmíme pozorovat závislost mezi fitovanými hodnotami a residuály, jak bylo popsáno v kapitole \ref{chap:lme}. V našem modelu není žádná výrazná heteroskedasticita, tedy není zde zřejmý vztah mezi residuály modelu a fitovanými hodnotami, viz obrázek \ref{fig:resvsfit_curt}.

\begin{figure}
	\centering
  \includegraphics[width=0.55\textwidth]{img/ch2/modmax15cm_curt.png}
	\caption{Srovnání residuálů modelu s fitovanými hodnotami pro transformaci pomocí třetí odmocniny.}
	\label{fig:resvsfit_curt}
\end{figure}

Pro ilustraci důležitosti ARMA modelu na obrázku \ref{fig:acf_curtnoARMA} autokorelační funkci bez modelu ARMA a na \ref{fig:acf_curtARMA22} s modelem ARMA. Hodnota $\text{ACF}$ pro $\text{lag}=0$ je vyřazená pro větší přehlednost grafu, vždy nabývá hodnoty $1$.

Vidíme, že přidání ARMA s hodnotami ($p=2$ a $q=1$) se výrazně zlepší autokorelační funkce a korelační strukturu jsme téměř odfiltrovali. Testovali jsme i jiné hodnoty $p$ a $q$, ale pro $p=2$ a $q=1$ nám vyšla autokorelační funkce s nejmenšími hodnotami. S rostoucím $p$ a $q$ také roste výpočetní náročnost, pro $p=3$ a $q=3$ se ACF téměř nezmění, ale výpočet trvá až 4-krát déle (na počítači, který byl používán ke zpracování, šlo o více než 3 hodiny výpočetního času).

\begin{figure}
	\centering
	\begin{subfigure}{0.45\textwidth}
  \includegraphics[width=\textwidth]{img/ch2/acf_curt.png}
		\caption{Autokorelační funkce pro model s transformací $\sqrt[3]{}$, ale bez modelování autokorelační struktury.}
		\label{fig:acf_curtnoARMA}
	\end{subfigure}
	\hfill
	\begin{subfigure}{0.45\textwidth}
  \includegraphics[width=\textwidth]{img/ch2/acf_curtARMA22.png}
		\caption{Autokorelační funkce pro model s transformací $\sqrt[3]{}$, s modelování autokorelační struktury, kde $p=2,\ q=1$.}
		\label{fig:acf_curtARMA22}
	\end{subfigure}
	\caption{Srovnání autokorelační funkce modelu s a bez ARMA}
	\label{fig:acf_curt}
\end{figure}

\subsection{Seznam modelů}
Výše jsme ukázali, jakým způsobem zpracováváme jednotlivé modely. Pro přehlednost označujeme modely zkratkou. Seznam modelů je v tabulce \ref{tab:seznammodelu}. Kromě již zmiňovaných modelů pro maximální a minimální denní teploty a výšku $\SI{15}{cm}$ a $\SI{0}{cm}$ jsme počítali modely pouze pro období od začátku května do konce října (teplé období) a pro období od začátku listopadu do konce dubna (studené období). Pro modely s čidlem ve výšce $\SI{15}{cm}$ ve studeném období jsme spočetli modely, kdy jsme výšku sněhu nahradili kategorickou proměnou ($0$ odpovídá hodnotám bez sněhu, $1$ čidlo je nad sněhem, $2$ čidlo je pod sněhem). K těmto 16 modelům jsme přidali ještě dalších 16 modelů, které mají stejné parametry, ale místo rozdílu teplot např. $\Delta t = t_{15cm} - t_{2m}$ jako závislou proměnnou bereme její absolutní hodnotu, jako např. $\Delta t_{abs} = \left|t_{15cm} - t_{2m}\right|$. Tyto modely budeme označovat písmenem "A" na začátku (např. AMax15all) a pomůžou nám v interpretaci výsledků.

\begin{table}
\centering\footnotesize\sf
\begin{tabular}{lrrrrr}
\toprule
	Název modelu & Teploty & Výška čidla & Období & Počet čidel & Počet měření \\
\midrule
	Max15all & max. & $\SI{15}{cm}$ & vše & $157$ & $76627$ \\
	Max0all & max. & $\SI{0}{cm}$ & vše & $157$ & $76635$ \\
	Min15all & min. & $\SI{15}{cm}$ & vše & $157$ & $74225$ \\
	Min0all & min. & $\SI{0}{cm}$ & vše & $157$ & $74083$ \\
	Max15warm & max. & $\SI{15}{cm}$ & teplé & $157$ & $32062$ \\
	Max0warm & max. & $\SI{0}{cm}$ & teplé & $157$ & $32108$ \\
	Min15warm & min. & $\SI{15}{cm}$ & teplé & $157$ & $30716$ \\
	Min0warm & min. & $\SI{0}{cm}$ & teplé & $157$ & $30651$ \\
	Max15cold & max. & $\SI{15}{cm}$ & studené & $156$ & $44563$ \\
	Max0cold & max. & $\SI{0}{cm}$ & studené & $156$ & $44528$ \\
	Min15cold & min. & $\SI{15}{cm}$ & studené & $156$ & $43505$ \\
	Min0cold & min. & $\SI{0}{cm}$ & studené & $156$ & $43429$ \\
	Max15allc & max. & $\SI{15}{cm}$ & vše & $157$ & $76627$ \\
	Max15coldc & max. & $\SI{15}{cm}$ & studené & $156$ & $44563$ \\
	Min15allc & min. & $\SI{15}{cm}$ & vše & $157$ & $74225$ \\
	Min15coldc & min. & $\SI{15}{cm}$ & studené & $156$ & $43505$ \\
\bottomrule
\end{tabular}
	\caption{Seznam lineárních smíšených modelů pro vyhodnocení vlivu meteorologických podmínek na rozdíl teplot. "c" u posledních čtyř modelů znamená, že prediktor výšky sněhu je nahrazen kategoriemi (bez sněhu, nad sněhem a pod sněhem). Druhou sadu modelů, kdy bereme absolutní hodnotu závislé proměnné, tak budeme značit písmenem "A" na začátku názvu.}
	\label{tab:seznammodelu}
\end{table}

\subsection{Použitý software}
Pro zpracování dat jsme využívali primárně programovací jazyk \texttt{R} \parencite{Rlanguage} a jeho verzi \texttt{4.1.3}. Použili jsme následující baličky a knihovny pro přípravu a analýzu dat: \texttt{stringr} \parencite{stringr}, \texttt{e1071} \parencite{e1071}, \texttt{xlsx} \parencite{xlsx}, \texttt{geosphere} \parencite{geosphere}, \texttt{ggplot2} \parencite{ggplot2}, \texttt{lmtest} \parencite{lmtest}, \texttt{data.table} \parencite{data.table}, \texttt{lubridate} \parencite{lubridate}, \texttt{nlme} \parencite{nlme}, \texttt{forecast} \parencite{forecast}, \texttt{moments} \parencite{moments}, \texttt{profvis} \parencite{profvis} a \texttt{climate} \parencite{climate}. Pro otevření dat formátu \texttt{grib} a jejich transformaci z reanalýzy ERA5 jsme použili programovací jazyk \texttt{Python}, verzi \texttt{3.8.13} \parencite{python} a knihovny \texttt{xarray} \parencite{xarray}, \texttt{numpy} \parencite{numpy}, \texttt{pandas} \parencite{pandas}, \texttt{cfgrib} \parencite{cfgrib} a \texttt{eccodes} \parencite{eccodes}. Pro vykreslení obrázku \ref{fig:rozlozenicidel} byl použit software \texttt{QGIS}, verze \texttt{3.10} \parencite{qgis}
